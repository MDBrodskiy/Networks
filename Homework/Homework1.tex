%%%%%%%%%%%%%%%%%%%%%%%%%%%%%%%%%%%%%%%%%%%%%%%%%%%%%%%%%%%%%%%%%%%%%%%%%%%%%%%%%%%%%%%%%%%%%%%%%%%%%%%%%%%%%%%%%%%%%%%%%%%%%%%%%%%%%%%%%%%%%%%%%%%%%%%%%%%%%%%%%%%
% Written By Michael Brodskiy
% Class: Fundamentals of Networks
% Professor: E. Bernal Mor
%%%%%%%%%%%%%%%%%%%%%%%%%%%%%%%%%%%%%%%%%%%%%%%%%%%%%%%%%%%%%%%%%%%%%%%%%%%%%%%%%%%%%%%%%%%%%%%%%%%%%%%%%%%%%%%%%%%%%%%%%%%%%%%%%%%%%%%%%%%%%%%%%%%%%%%%%%%%%%%%%%%

\documentclass[12pt]{article} 
\usepackage{alphalph}
\usepackage[utf8]{inputenc}
\usepackage[russian,english]{babel}
\usepackage{titling}
\usepackage{amsmath}
\usepackage{graphicx}
\usepackage{enumitem}
\usepackage{amssymb}
\usepackage[super]{nth}
\usepackage{everysel}
\usepackage{ragged2e}
\usepackage{geometry}
\usepackage{multicol}
\usepackage{fancyhdr}
\usepackage{cancel}
\usepackage{siunitx}
\usepackage{physics}
\usepackage{tikz}
\usepackage{mathdots}
\usepackage{yhmath}
\usepackage{cancel}
\usepackage{color}
\usepackage{array}
\usepackage{multirow}
\usepackage{gensymb}
\usepackage{tabularx}
\usepackage{extarrows}
\usepackage{booktabs}
\usepackage{lastpage}
\usepackage{float}
\usetikzlibrary{fadings}
\usetikzlibrary{patterns}
\usetikzlibrary{shadows.blur}
\usetikzlibrary{shapes}

\geometry{top=1.0in,bottom=1.0in,left=1.0in,right=1.0in}
\newcommand{\subtitle}[1]{%
  \posttitle{%
    \par\end{center}
    \begin{center}\large#1\end{center}
    \vskip0.5em}%

}
\usepackage{hyperref}
\hypersetup{
colorlinks=true,
linkcolor=blue,
filecolor=magenta,      
urlcolor=blue,
citecolor=blue,
}


\title{Conceptual Homework 1}
\date{October 2, 2023}
\author{Michael Brodskiy\\ \small Professor: E. Bernal Mor}

\begin{document}

\maketitle

\begin{enumerate}

  \item Consider sending $L$-byte packets from a source host to a destination host over a fixed route. List the delay components in the end-to-end delay of a packet. Which of these delays are constant and which are variable?

    First and foremost, we know that the packets consist of $L$-bytes, but we do not know how many nodes there are between the host and destination. The total delay would be expressed as the sum of nodal delays:

    $$\sum_{\text{host}}^{\text{dest}}d_{nodal}$$

    Each nodal delay consists of four delay types:

    \begin{itemize}

      \item Processing Delay

        The processing delay is pretty much insignificant in modern routers, as it is on the order of microseconds. Thus, we can treat this delay as constant (zero).

      \item Queueing Delay

        The queueing delay depends on the congestion level of a router, and, thus, can vary. Every output link has its own queue, and, because buffers have limited memory space, the arrival rate may sometimes exceed the transmission rate.

      \item Transmission Delay

        Transmission delay, arguably the most important of the four delays, depends on the packet size and transmission rate, and, thus, is variable. Since there are most likely multiple nodes from source to destination, the transmission rate will vary, causing the delay to vary.

      \item Propagation Delay

        The propagation delay depends on the physical link length, as well as the speed of propagation. Specifically in this scenario, since there is a fixed route, both of these will remain constant, which keeps the propagation delay constant; however, in general, the propagation delay varies depending on from where and to where a transmission occurs.

    \end{itemize}
    
  \item How long is the propagation delay for a packet of length $1,000$ bytes over a link of distance $2,500$ km, propagation speed $2.5\cdot10^8$ m/s, and transmission rate $2$ Mbps? More generally, how long is the propagation delay for a packet of length $L$ over a link of distance $d$, propagation speed $s$, and transmission rate $R$ bps? Does this delay depend on packet length?  Does this delay depend on transmission rate?

    In this case, the propagation delay is:

    $$\frac{2,500,000[\si{\meter}]}{2.5\cdot10^{8}\left[ \frac{\si{\meter}}{\si{\second}} \right]}=.01[\si{\second}]$$

    In general, the propagation delay may be represented as a ratio of physical link length to propagation speed:

    $$d_{prop}=\frac{d}{s}$$

    The delay does \underline{not} depend on packet length or transmission rate

  \item Suppose Host A wants to send a large file to Host B. The path from Host A to Host B has three links, of rates $R_1=500$ kbps, $R_2=2$ Mbps, and $R_3=1$ Mbps. Assume no other traffic in the network.

    \begin{enumerate}

      \item What is the average throughput for the file transfer?

        To find the average throughput, we simply have to average out our values, assuming we take Mbps as the unit:

        $$\frac{.5 + 2 + 1}{3}=1[\text{Mbps}]$$

      \item Suppose the file is 4 million bytes. The first bit arrives at the receiver at a given time. From this time, how long will it take to receive all the file in Host B approximately?

        $$4,000,000[\text{bytes}]=4,000[\text{kb}]=4[\text{Mb}]$$
        $$\frac{4}{1}=4[\si{\second}]$$

      \item Repeat (a) and (b), but now with $R_2$ reduced to $100$ kbps.

        The throughput is:

        $$\frac{.5+.1+1}{3}=.533\bar{3}[\text{Mbps}]$$

        This makes the time to file receipt:

        $$\frac{4}{.533\bar{3}}=7.5[\si{\second}]$$

    \end{enumerate}
    
  \item Considering only transmission delays, the equation for the end-to-end transmission delay, $D$, of sending one packet of length $L$ over $N$ links of transmission rate $R$ is $D=N(L/R)$. Generalize this formula for sending $P$ such packets back-to-back over the $N$ links.

    We know that the first packet will always arrive at:

    $$t_1=\frac{NL}{R}$$

    If each packet were numbered, it would arrive at some time:

    $$t_n=\frac{nL}{R}+t_1$$

    Because each the $n$ term is initially zero, we can rewrite this as:

    $$D_n=\frac{NL}{R}+\frac{(P-1)L}{R}=(N+P-1)\frac{L}{R}$$
    
  \item Do a quick search and list 5 nonproprietary Internet applications and the application-layer protocols that they use.

    \begin{enumerate}

      \item Jitsi Meet — Session Initiation Protocol (SIP)

      \item Neomutt/mutt — POP3/IMAP/SMTP

      \item ProtonVPN — OpenVPN + TLS/OpenSSL

      \item Tor — Onion + TCP

      \item Qutebrowser — HTTP/S + OCSP + WebRTC + WebSocket

    \end{enumerate}
    
  \item For a communication session between a pair of processes, which process is the client, and which is the server? In P2P architecture, can a peer run a server process? Briefly explain.

    In a client-server configuration, one may differentiate between a client and a server in the following ways:

    \begin{itemize}

      \item Servers — Servers have permanent IP addresses, are always on, and are generally present in in data centers for ease of scaling

      \item Clients — Communicate with the server, may connect intermittently, may have dynamic IP addresses, and do not communicate with each other

    \end{itemize}

    In a P2P architecture, peers can technically run a server process, as peers take on the role of both a client in the server. This occurs because, at the same time as peers distribute files and receive connections, they also connect and receive from other peers.
    
  \item What information is used by a process running on one host to identify a process running on another host?

    An IP address allows for the identification of a host; the port number identifies a process running on said host. For example, the port number 80 is generally used for HTTP.
    
  \item Suppose you wanted to do a short transaction from a remote client to a server as fast as possible. Would you use UDP or TCP? Why?

    Given that the transaction needs to be short and fast, the User Datagram Protocol (UDP) would be more apt for this situation. Since speed is a priority, the lack of reliability, as well as congestion and flow control would speed up the transfer; however, the transaction would not be guaranteed, as UDP would not recover from the packet loss.
    
  \item Consider the following string of ASCII characters that were captured by Wireshark when the browser sent an HTTP GET message (\textit{i}.\textit{e}., this is the actual content of an HTTP GET message). The \textsc{<cr><lf>} represents the carriage return and line-feed characters.

    \begin{flushleft}
      \textsc{GET \textcolor{red}{/cs453/index.html} \textcolor{green}{HTTP/1.1}<cr><lf>Host: \textcolor{red}{gaia.cs.umass.edu}<cr><lf>\textcolor{purple}{User-Agent: Mozilla/5.0 (Windows;U; Windows NT 5.1; en-US; rv:1.7.2)} Gecko/20040804 Netscape/7.2 (ax) <cr><lf>Accept:ext/xml, application/xml, application/xhtml+xml, text/html;q=0.9, text/plain;q=0.8, image/png,*/*;q=0.5<cr><lf>Accept-Language: en-us, en;q=0.5<cr><lf>Accept-Encoding: zip, deflate<cr><lf>Accept-Charset: ISO-8859-1, utf-8;q=0.7,*;q=0.7<cr><lf>\textcolor{blue}{Keep-Alive: 300}<cr><lf>\textcolor{blue}{Connection:keep-alive}<cr><lf><cr><lf>}
    \end{flushleft}

    Answer the following questions, indicating where in the HTTP GET message above you find the answer.

    \begin{enumerate}

      \item What is the complete URL of the document requested by the browser?

        The host is represented by the \textsc{gaia.cs.umass.edu} and the specific webpage is \textsc{/cs453/index.html}. Combining these two together, we find that the full URL is \textsc{gaia.cs.umass.edu/cs453/index.html}. This is exhibited in the portions in \textcolor{red}{red text}.

      \item What version of HTTP is the browser running?

        The browser is running HTTP/1.1. This is displayed in the portion in \textcolor{green}{green text}.

      \item Does the browser request a non-persistent or a persistent connection?

        The browser requests a persistent connection. This is evident with the \textsc{keep-alive} connection request, in addition to the time specified to keep the connection alive (300 seconds). The pertinent portion is in \textcolor{blue}{blue text} above.

      \item What type of browser initiates this message? Why is the browser type needed in a HTTP request message?

        The browser is most likely Mozilla Firefox, indicated by the user-agent above. Furthermore, the operating system can be seen as Windows. The relevant portion is in \textcolor{purple}{purple text}.

    \end{enumerate}
    
\end{enumerate}

\end{document}

