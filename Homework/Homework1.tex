%%%%%%%%%%%%%%%%%%%%%%%%%%%%%%%%%%%%%%%%%%%%%%%%%%%%%%%%%%%%%%%%%%%%%%%%%%%%%%%%%%%%%%%%%%%%%%%%%%%%%%%%%%%%%%%%%%%%%%%%%%%%%%%%%%%%%%%%%%%%%%%%%%%%%%%%%%%%%%%%%%%
% Written By Michael Brodskiy
% Class: Fundamentals of Networks
% Professor: E. Bernal Mor
%%%%%%%%%%%%%%%%%%%%%%%%%%%%%%%%%%%%%%%%%%%%%%%%%%%%%%%%%%%%%%%%%%%%%%%%%%%%%%%%%%%%%%%%%%%%%%%%%%%%%%%%%%%%%%%%%%%%%%%%%%%%%%%%%%%%%%%%%%%%%%%%%%%%%%%%%%%%%%%%%%%

\documentclass[12pt]{article} 
\usepackage{alphalph}
\usepackage[utf8]{inputenc}
\usepackage[russian,english]{babel}
\usepackage{titling}
\usepackage{amsmath}
\usepackage{graphicx}
\usepackage{enumitem}
\usepackage{amssymb}
\usepackage[super]{nth}
\usepackage{everysel}
\usepackage{ragged2e}
\usepackage{geometry}
\usepackage{multicol}
\usepackage{fancyhdr}
\usepackage{cancel}
\usepackage{siunitx}
\usepackage{physics}
\usepackage{tikz}
\usepackage{mathdots}
\usepackage{yhmath}
\usepackage{cancel}
\usepackage{color}
\usepackage{array}
\usepackage{multirow}
\usepackage{gensymb}
\usepackage{tabularx}
\usepackage{extarrows}
\usepackage{booktabs}
\usepackage{lastpage}
\usepackage{float}
\usetikzlibrary{fadings}
\usetikzlibrary{patterns}
\usetikzlibrary{shadows.blur}
\usetikzlibrary{shapes}

\geometry{top=1.0in,bottom=1.0in,left=1.0in,right=1.0in}
\newcommand{\subtitle}[1]{%
  \posttitle{%
    \par\end{center}
    \begin{center}\large#1\end{center}
    \vskip0.5em}%

}
\usepackage{hyperref}
\hypersetup{
colorlinks=true,
linkcolor=blue,
filecolor=magenta,      
urlcolor=blue,
citecolor=blue,
}


\title{Conceptual Homework 1}
\date{October 2, 2023}
\author{Michael Brodskiy\\ \small Professor: E. Bernal Mor}

\begin{document}

\maketitle

\begin{enumerate}

  \item Consider sending $L$-byte packets from a source host to a destination host over a fixed route. List the delay components in the end-to-end delay of a packet. Which of these delays are constant and which are variable?
    
  \item How long is the propagation delay for a packet of length $1,000$ bytes over a link of distance $2,500$ km, propagation speed $2.5\cdot10^8$ m/s, and transmission rate $2$ Mbps? More generally, how long is the propagation delay for a packet of length $L$ over a link of distance $d$, propagation speed $s$, and transmission rate $R$ bps? Does this delay depend on packet length?  Does this delay depend on transmission rate?
    
  \item Suppose Host A wants to send a large file to Host B. The path from Host A to Host B has three links, of rates $R_1=500$ kbps, $R_2=2$ Mbps, and $R_3=1$ Mbps. Assume no other traffic in the network.

    \begin{enumerate}

      \item What is the average throughput for the file transfer?

      \item Suppose the file is 4 million bytes. The first bit arrives at the receiver at a given time. From this time, how long will it take to receive all the file in Host B approximately?

      \item Repeat (a) and (b), but now with $R_2$ reduced to $100$ kbps.

    \end{enumerate}
    
  \item Considering only transmission delays, the equation for the end-to-end transmission delay, $D$, of sending one packet of length $L$ over $N$ links of transmission rate $R$ is $D=N(L/R)$. Generalize this formula for sending $P$ such packets back-to-back over the $N$ links.
    
  \item Do a quick search and list 5 nonproprietary Internet applications and the application-layer protocols that they use.
    
  \item For a communication session between a pair of processes, which process is the client, and which is the server? In P2P architecture, can a peer run a server process? Briefly explain.
    
  \item What information is used by a process running on one host to identify a process running on another host?

    An IP address allows for the identification of a host; the port number identifies a process running on said host. For example, the port number 80 is generally used for HTTP.
    
  \item Suppose you wanted to do a short transaction from a remote client to a server as fast as possible. Would you use UDP or TCP? Why?

    Given that the transaction needs to be short and fast, the User Datagram Protocol (UDP) would be more apt for this situation. Since speed is a priority, the lack of reliability, as well as congestion and flow control would speed up the transfer; however, the transaction would not be guaranteed, as UDP would not recover from the packet loss.
    
  \item Consider the following string of ASCII characters that were captured by Wireshark when the browser sent an HTTP GET message (\textit{i}.\textit{e}., this is the actual content of an HTTP GET message). The \textsc{<cr><lf>} represents the carriage return and line-feed characters.

    \begin{flushleft}
      \textsc{GET \textcolor{red}{/cs453/index.html} \textcolor{green}{HTTP/1.1}<cr><lf>Host: \textcolor{red}{gaia.cs.umass.edu}<cr><lf>\textcolor{purple}{User-Agent: Mozilla/5.0 (Windows;U; Windows NT 5.1; en-US; rv:1.7.2)} Gecko/20040804 Netscape/7.2 (ax) <cr><lf>Accept:ext/xml, application/xml, application/xhtml+xml, text/html;q=0.9, text/plain;q=0.8, image/png,*/*;q=0.5<cr><lf>Accept-Language: en-us, en;q=0.5<cr><lf>Accept-Encoding: zip, deflate<cr><lf>Accept-Charset: ISO-8859-1, utf-8;q=0.7,*;q=0.7<cr><lf>\textcolor{blue}{Keep-Alive: 300}<cr><lf>\textcolor{blue}{Connection:keep-alive}<cr><lf><cr><lf>}
    \end{flushleft}

    Answer the following questions, indicating where in the HTTP GET message above you find the answer.

    \begin{enumerate}

      \item What is the complete URL of the document requested by the browser?

        The host is represented by the \textsc{gaia.cs.umass.edu} and the specific webpage is \textsc{/cs453/index.html}. Combining these two together, we find that the full URL is \textsc{gaia.cs.umass.edu/cs453/index.html}. This is exhibited in the portions in \textcolor{red}{red text}.

      \item What version of HTTP is the browser running?

        The browser is running HTTP/1.1. This is displayed in the portion in \textcolor{green}{green text}.

      \item Does the browser request a non-persistent or a persistent connection?

        The browser requests a persistent connection. This is evident with the \textsc{keep-alive} connection request, in addition to the time specified to keep the connection alive (300 seconds). The pertinent portion is in \textcolor{blue}{blue text} above.

      \item What type of browser initiates this message? Why is the browser type needed in a HTTP request message?

        The browser is most likely Mozilla Firefox, indicated by the user-agent above. Furthermore, the operating system can be seen as Windows. The relevant portion is in \textcolor{purple}{purple text}.

    \end{enumerate}
    
\end{enumerate}

\end{document}

