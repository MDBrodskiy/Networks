%%%%%%%%%%%%%%%%%%%%%%%%%%%%%%%%%%%%%%%%%%%%%%%%%%%%%%%%%%%%%%%%%%%%%%%%%%%%%%%%%%%%%%%%%%%%%%%%%%%%%%%%%%%%%%%%%%%%%%%%%%%%%%%%%%%%%%%%%%%%%%%%%%%%%%%%%%%%%%%%%%%
% Written By Michael Brodskiy
% Class: Fundamentals of Networks
% Professor: E. Bernal Mor
%%%%%%%%%%%%%%%%%%%%%%%%%%%%%%%%%%%%%%%%%%%%%%%%%%%%%%%%%%%%%%%%%%%%%%%%%%%%%%%%%%%%%%%%%%%%%%%%%%%%%%%%%%%%%%%%%%%%%%%%%%%%%%%%%%%%%%%%%%%%%%%%%%%%%%%%%%%%%%%%%%%

\documentclass[12pt]{article} 
\usepackage{alphalph}
\usepackage[utf8]{inputenc}
\usepackage[russian,english]{babel}
\usepackage{titling}
\usepackage{amsmath}
\usepackage{graphicx}
\usepackage{enumitem}
\usepackage{amssymb}
\usepackage[super]{nth}
\usepackage{everysel}
\usepackage{ragged2e}
\usepackage{geometry}
\usepackage{multicol}
\usepackage{fancyhdr}
\usepackage{cancel}
\usepackage{siunitx}
\usepackage{physics}
\usepackage{tikz}
\usepackage{mathdots}
\usepackage{yhmath}
\usepackage{cancel}
\usepackage{color}
\usepackage{array}
\usepackage{multirow}
\usepackage{gensymb}
\usepackage{tabularx}
\usepackage{extarrows}
\usepackage{booktabs}
\usepackage{lastpage}
\usepackage{float}
\usetikzlibrary{fadings}
\usetikzlibrary{patterns}
\usetikzlibrary{shadows.blur}
\usetikzlibrary{shapes}

\geometry{top=1.0in,bottom=1.0in,left=1.0in,right=1.0in}
\newcommand{\subtitle}[1]{%
  \posttitle{%
    \par\end{center}
    \begin{center}\large#1\end{center}
    \vskip0.5em}%

}
\usepackage{hyperref}
\hypersetup{
colorlinks=true,
linkcolor=blue,
filecolor=magenta,      
urlcolor=blue,
citecolor=blue,
}


\title{The Network Layer}
\date{\today}
\author{Michael Brodskiy\\ \small Professor: E. Bernal Mor}

\begin{document}

\maketitle

\begin{itemize}

  \item Network Layer Overview

    \begin{itemize}

      \item Transport segment from sending to receiving host

        \begin{itemize}

          \item Sender: encapsulates segments into packets, passes to link layer

          \item Receiver: extracts segments from packets and delivers segments to transport layer protocol

        \end{itemize}

    \end{itemize}

  \item Network Layer Functions

    \begin{itemize}

      \item Forwarding: move packets from router's input link to appropriate router's output link

      \item Routing: determine route taken by packets from source to destination

        \begin{itemize}

          \item Routing Algorithms

        \end{itemize}

      \item Analogy: Taking a Trip

        \begin{itemize}

          \item Forwarding: process of getting through single intersection

          \item Routing: process of planning trip from source to destination

        \end{itemize}

    \end{itemize}

  \item Data Plane

    \begin{itemize}

      \item Local, per-router function

      \item Determines hoe packet arriving on router input port is forwarded to router output port

    \end{itemize}

  \item Control Plane

    \begin{itemize}

      \item Network-wide logic

      \item Determines how packet is routed among routers along end-end path from source host to destination host

      \item Two control-plane approaches

        \begin{itemize}

          \item Traditional routing algorithms: implemented in routers

          \item Software-Defined Networking (SDN): implemented in (remote) servers

        \end{itemize}

    \end{itemize}

  \item Traditional Control Plane Algorithms

    \begin{itemize}

      \item Individual routing algorithm components in each and every router interact in the control plane

    \end{itemize}

  \item SDN Control Plane

    \begin{itemize}

      \item Remote controller interacts with local Control Agents (CAs) to compute, install forwarding tables in routers

    \end{itemize}

  \item Network Layer Service Model

    \begin{itemize}

      \item A network layer service model defines the characteristics of end-to-end transport of packets between sending and receiving hosts

      \item Examples of possible services (this is only a partial list, there are countless variants):

        \begin{itemize}

          \item Guaranteed delivery

          \item Guaranteed delivery with bounded delay

          \item In-order packet delivery

          \item Guaranteed minimum transmission rate

          \item Security

        \end{itemize}

      \item Services provided by the network layer: two main options

        \begin{enumerate}

          \item Connection-oriented service

            \begin{itemize}

              \item A path from source all the way to destination must be established before any data packets can be sent

                \begin{itemize}

                  \item This connection is called a Virtual Circuit (VC)

                  \item The network is called a virtual-circuit network

                  \item Each VC requires router table space and reservation of resources

                \end{itemize}

              \item Designed to provide some quality of service (QoS) (\textit{i}.\textit{e}.\ maximum delay guarantees, minimum losses, minimum throughput guarantees, etc.)

              \item Example: Asynchronous Transfer Mode (ATM) $\to$ popular in the 90s early 200, being replaced by all-IP architecyres

            \end{itemize}

          \item Connectionless service

            \begin{itemize}

              \item Best-effort service

              \item Packets are injected into the network individually and routed independently of each other

              \item No advance setup is needed

              \item No error or flow service functionalities provided

                \begin{itemize}

                  \item The transport layer might do something end-to-end

                  \item The link layer might do something at the link level

                \end{itemize}

              \item For example, IP (internet protocol)

            \end{itemize}

        \end{enumerate}

    \end{itemize}
    
  \item Reflections on Best-Effort Service

    \begin{itemize}

      \item Simplicity of mechanism has allowed Internet to be widely deployed and adopted

      \item Sufficient provisioning of capacity allows performance of real-time applications (\textit{e}.\textit{g}.\ interactive voice, video) to be ``good enough'' for ``most of the time''

      \item Replicated, application-layer distributed services (data centers, content distribution networks) connecting close to clients' networks, allow services to be provided from multiple locations

      \item Congestion control at the transport layer of ``elastic'' services helps

    \end{itemize}

  \item Input Ports

    \begin{itemize}

      \item Decentralized Switching:

        \begin{itemize}

          \item Using header field values, lookup output port using forwarding table in input port memory (``match plus action'')

            \begin{itemize}

              \item Destination-based forwarding: forward based only on destination IP address (traditional)

              \item Generalized forwarding: forward based on any set of header field values

              \item Input port queueing: if packets arrive faster than forwarding rate into switch fabric

            \end{itemize}

        \end{itemize}

    \end{itemize}

  \item Input Port Queueing

    \begin{itemize}

      \item If switch fabric slower than input ports combined $\to$ queueing may occur at input queues

        \begin{itemize}

          \item Queueing delay and loss due to input buffer overflow

        \end{itemize}

      \item Head-of-the-Line (HOL) blocking: queued packet at front of queue prevents others in queue from moving forward

    \end{itemize}

  \item Output Ports

    \begin{itemize}

      \item Buffering required when packets arrive from fabric faster than link transmission rate

      \item Drop policy: which packets to drop if no free buffers?

      \item Scheduling discipline chooses among queued packets for next transmission

        \begin{itemize}

          \item FCFS (First Come, First Served), priority, \ldots

        \end{itemize}

    \end{itemize}

  \item The Internet Protocol

    \begin{itemize}

      \item The glue that holds the whole Internet together (data plane)

        \begin{itemize}

          \item Designed with internetworking in mind

        \end{itemize}

      \item Provides a best-effort (no guaranteee) way to transport IP packets (aka datagrams) from source to destination

        \begin{itemize}

          \item Without regard to whether these machines are on the same network or whether there are other networks between them

        \end{itemize}

      \item There are two versions of IP in use today

        \begin{itemize}

          \item IPv4 (IP version 4)

            \begin{itemize}

              \item The first ``major version'' of IP and currently the dominant protocol of the Internet

            \end{itemize}

          \item IPv6

        \end{itemize}

    \end{itemize}

  \item IP Fragmentation

    \begin{itemize}

      \item Network links have MTU (maximum transmission unit)

        \begin{itemize}

          \item MTU: largest possible payload in link-level frame $\to$ maximum IP packet size

          \item Different link types, different MTUs

        \end{itemize}

      \item Problem: IP packet larger than MTU of output link

        \begin{itemize}

          \item Solution: Fragmentation?

            \begin{itemize}

              \item Typically, IPv6 does not allow fragmentation

              \item Typically, TCP does not allow fragmentation

            \end{itemize}

        \end{itemize}

    \end{itemize}

  \item IP Alternative to Fragmentation

    \begin{itemize}

      \item If fragmentation is no allowed $\to$ ``path MTU discovery''

      \item Path MTU Discovery

        \begin{itemize}

          \item Each IPv4 packet is sent with its header bits set to indicate that fragmentation is not allowed to be performed (flag DF=1)

          \item Added start-up delat

          \item The transport layer can learn about the MTU to adapt the Maximum Segment Size (MSS)

        \end{itemize}

    \end{itemize}

  \item IP Addressing: Introduction

    \begin{itemize}

      \item IPv4 Address: 32-bit identifier associated with each host or router interface

      \item Interface: connectio between host/router and physical link

        \begin{itemize}

          \item Router's typically have multiple interfaces

          \item Host typically has one or two interfaces (e.g, wired Ethernet, wireless 802.11)

        \end{itemize}

    \end{itemize}

  \item Subnets

    \begin{itemize}

      \item Device interfaces that can physically reach each other without passing through an intervening router

      \item IP Addresses have structure:

        \begin{itemize}

          \item Network portion (aka subnet portion): high order bits

            \begin{itemize}

              \item Devices in same subnet have common network portion

            \end{itemize}

          \item Host portion: remaining low order bits

        \end{itemize}

    \end{itemize}

  \item IP Addresssing in Subnets: CIDR

    \begin{itemize}

      \item CIDR: Classless Inter Domain Routing (pronouned ``cider'')

        \begin{itemize}

          \item Network portion (aka prefix) of address of arbitrary length

          \item Address format (by convention): \textsc{a.b.c.d.x}, where \textsc{x} os the number of bits in the network portion of the address

        \end{itemize}

      \item Network address (subnet address): network portion and 0s in the host portion/\textsc{x}

      \item Subnet mask: binary mask of 1s in teh subnet portion and 0s in the host portion $\to$ \textsc{x}

        \begin{itemize}

          \item The subnet mask can be ANDed with an IP address to obtain the network address

        \end{itemize}

      \item Recipe for identifying subnets

        \begin{itemize}

          \item Detach each interface from its host or router, creating ``islands'' of isolated networks

          \item Each isolated network is a subnet

        \end{itemize}

    \end{itemize}

\end{itemize}

\end{document}

