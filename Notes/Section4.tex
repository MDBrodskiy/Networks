%%%%%%%%%%%%%%%%%%%%%%%%%%%%%%%%%%%%%%%%%%%%%%%%%%%%%%%%%%%%%%%%%%%%%%%%%%%%%%%%%%%%%%%%%%%%%%%%%%%%%%%%%%%%%%%%%%%%%%%%%%%%%%%%%%%%%%%%%%%%%%%%%%%%%%%%%%%%%%%%%%%
% Written By Michael Brodskiy
% Class: Fundamentals of Networks
% Professor: E. Bernal Mor
%%%%%%%%%%%%%%%%%%%%%%%%%%%%%%%%%%%%%%%%%%%%%%%%%%%%%%%%%%%%%%%%%%%%%%%%%%%%%%%%%%%%%%%%%%%%%%%%%%%%%%%%%%%%%%%%%%%%%%%%%%%%%%%%%%%%%%%%%%%%%%%%%%%%%%%%%%%%%%%%%%%

\documentclass[12pt]{article} 
\usepackage{alphalph}
\usepackage[utf8]{inputenc}
\usepackage[russian,english]{babel}
\usepackage{titling}
\usepackage{amsmath}
\usepackage{graphicx}
\usepackage{enumitem}
\usepackage{amssymb}
\usepackage[super]{nth}
\usepackage{everysel}
\usepackage{ragged2e}
\usepackage{geometry}
\usepackage{multicol}
\usepackage{fancyhdr}
\usepackage{cancel}
\usepackage{siunitx}
\usepackage{physics}
\usepackage{tikz}
\usepackage{mathdots}
\usepackage{yhmath}
\usepackage{cancel}
\usepackage{color}
\usepackage{array}
\usepackage{multirow}
\usepackage{gensymb}
\usepackage{tabularx}
\usepackage{extarrows}
\usepackage{booktabs}
\usepackage{lastpage}
\usepackage{float}
\usetikzlibrary{fadings}
\usetikzlibrary{patterns}
\usetikzlibrary{shadows.blur}
\usetikzlibrary{shapes}

\geometry{top=1.0in,bottom=1.0in,left=1.0in,right=1.0in}
\newcommand{\subtitle}[1]{%
  \posttitle{%
    \par\end{center}
    \begin{center}\large#1\end{center}
    \vskip0.5em}%

}
\usepackage{hyperref}
\hypersetup{
colorlinks=true,
linkcolor=blue,
filecolor=magenta,      
urlcolor=blue,
citecolor=blue,
}


\title{The Network Layer}
\date{\today}
\author{Michael Brodskiy\\ \small Professor: E. Bernal Mor}

\begin{document}

\maketitle

\begin{itemize}

  \item Network Layer Overview

    \begin{itemize}

      \item Transport segment from sending to receiving host

        \begin{itemize}

          \item Sender: encapsulates segments into packets, passes to link layer

          \item Receiver: extracts segments from packets and delivers segments to transport layer protocol

        \end{itemize}

    \end{itemize}

  \item Network Layer Functions

    \begin{itemize}

      \item Forwarding: move packets from router's input link to appropriate router's output link

      \item Routing: determine route taken by packets from source to destination

        \begin{itemize}

          \item Routing Algorithms

        \end{itemize}

      \item Analogy: Taking a Trip

        \begin{itemize}

          \item Forwarding: process of getting through single intersection

          \item Routing: process of planning trip from source to destination

        \end{itemize}

    \end{itemize}

  \item Data Plane

    \begin{itemize}

      \item Local, per-router function

      \item Determines hoe packet arriving on router input port is forwarded to router output port

    \end{itemize}

  \item Control Plane

    \begin{itemize}

      \item Network-wide logic

      \item Determines how packet is routed among routers along end-end path from source host to destination host

      \item Two control-plane approaches

        \begin{itemize}

          \item Traditional routing algorithms: implemented in routers

          \item Software-Defined Networking (SDN): implemented in (remote) servers

        \end{itemize}

    \end{itemize}

  \item Traditional Control Plane Algorithms

    \begin{itemize}

      \item Individual routing algorithm components in each and every router interact in the control plane

    \end{itemize}

  \item SDN Control Plane

    \begin{itemize}

      \item Remote controller interacts with local Control Agents (CAs) to compute, install forwarding tables in routers

    \end{itemize}

  \item Network Layer Service Model

    \begin{itemize}

      \item A network layer service model defines the characteristics of end-to-end transport of packets between sending and receiving hosts

      \item Examples of possible services (this is only a partial list, there are countless variants):

        \begin{itemize}

          \item Guaranteed delivery

          \item Guaranteed delivery with bounded delay

          \item In-order packet delivery

          \item Guaranteed minimum transmission rate

          \item Security

        \end{itemize}

      \item Services provided by the network layer: two main options

        \begin{enumerate}

          \item Connection-oriented service

            \begin{itemize}

              \item A path from source all the way to destination must be established before any data packets can be sent

                \begin{itemize}

                  \item This connection is called a Virtual Circuit (VC)

                  \item The network is called a virtual-circuit network

                  \item Each VC requires router table space and reservation of resources

                \end{itemize}

              \item Designed to provide some quality of service (QoS) (\textit{i}.\textit{e}.\ maximum delay guarantees, minimum losses, minimum throughput guarantees, etc.)

              \item Example: Asynchronous Transfer Mode (ATM) $\to$ popular in the 90s early 200, being replaced by all-IP architecyres

            \end{itemize}

          \item Connectionless service

            \begin{itemize}

              \item Best-effort service

              \item Packets are injected into the network individually and routed independently of each other

              \item No advance setup is needed

              \item No error or flow service functionalities provided

                \begin{itemize}

                  \item The transport layer might do something end-to-end

                  \item The link layer might do something at the link level

                \end{itemize}

              \item For example, IP (internet protocol)

            \end{itemize}

        \end{enumerate}

    \end{itemize}
    
  \item Reflections on Best-Effort Service

    \begin{itemize}

      \item Simplicity of mechanism has allowed Internet to be widely deployed and adopted

      \item Sufficient provisioning of capacity allows performance of real-time applications (\textit{e}.\textit{g}.\ interactive voice, video) to be ``good enough'' for ``most of the time''

      \item Replicated, application-layer distributed services (data centers, content distribution networks) connecting close to clients' networks, allow services to be provided from multiple locations

      \item Congestion control at the transport layer of ``elastic'' services helps

    \end{itemize}

  \item Input Ports

    \begin{itemize}

      \item Decentralized Switching:

        \begin{itemize}

          \item Using header field values, lookup output port using forwarding table in input port memory (``match plus action'')

            \begin{itemize}

              \item Destination-based forwarding: forward based only on destination IP address (traditional)

              \item Generalized forwarding: forward based on any set of header field values

              \item Input port queueing: if packets arrive faster than forwarding rate into switch fabric

            \end{itemize}

        \end{itemize}

    \end{itemize}

  \item Input Port Queueing

    \begin{itemize}

      \item If switch fabric slower than input ports combined $\to$ queueing may occur at input queues

        \begin{itemize}

          \item Queueing delay and loss due to input buffer overflow

        \end{itemize}

      \item Head-of-the-Line (HOL) blocking: queued packet at front of queue prevents others in queue from moving forward

    \end{itemize}

  \item Output Ports

    \begin{itemize}

      \item Buffering required when packets arrive from fabric faster than link transmission rate

      \item Drop policy: which packets to drop if no free buffers?

      \item Scheduling discipline chooses among queued packets for next transmission

        \begin{itemize}

          \item FCFS (First Come, First Served), priority, \ldots

        \end{itemize}

    \end{itemize}

  \item The Internet Protocol

    \begin{itemize}

      \item The glue that holds the whole Internet together (data plane)

        \begin{itemize}

          \item Designed with internetworking in mind

        \end{itemize}

      \item Provides a best-effort (no guaranteee) way to transport IP packets (aka datagrams) from source to destination

        \begin{itemize}

          \item Without regard to whether these machines are on the same network or whether there are other networks between them

        \end{itemize}

      \item There are two versions of IP in use today

        \begin{itemize}

          \item IPv4 (IP version 4)

            \begin{itemize}

              \item The first ``major version'' of IP and currently the dominant protocol of the Internet

            \end{itemize}

          \item IPv6

        \end{itemize}

    \end{itemize}

  \item IP Fragmentation

    \begin{itemize}

      \item Network links have MTU (maximum transmission unit)

        \begin{itemize}

          \item MTU: largest possible payload in link-level frame $\to$ maximum IP packet size

          \item Different link types, different MTUs

        \end{itemize}

      \item Problem: IP packet larger than MTU of output link

        \begin{itemize}

          \item Solution: Fragmentation?

            \begin{itemize}

              \item Typically, IPv6 does not allow fragmentation

              \item Typically, TCP does not allow fragmentation

            \end{itemize}

        \end{itemize}

    \end{itemize}

  \item IP Alternative to Fragmentation

    \begin{itemize}

      \item If fragmentation is no allowed $\to$ ``path MTU discovery''

      \item Path MTU Discovery

        \begin{itemize}

          \item Each IPv4 packet is sent with its header bits set to indicate that fragmentation is not allowed to be performed (flag DF=1)

          \item Added start-up delat

          \item The transport layer can learn about the MTU to adapt the Maximum Segment Size (MSS)

        \end{itemize}

    \end{itemize}

  \item IP Addressing: Introduction

    \begin{itemize}

      \item IPv4 Address: 32-bit identifier associated with each host or router interface

      \item Interface: connectio between host/router and physical link

        \begin{itemize}

          \item Router's typically have multiple interfaces

          \item Host typically has one or two interfaces (e.g, wired Ethernet, wireless 802.11)

        \end{itemize}

    \end{itemize}

  \item Subnets

    \begin{itemize}

      \item Device interfaces that can physically reach each other without passing through an intervening router

      \item IP Addresses have structure:

        \begin{itemize}

          \item Network portion (aka subnet portion): high order bits

            \begin{itemize}

              \item Devices in same subnet have common network portion

            \end{itemize}

          \item Host portion: remaining low order bits

        \end{itemize}

    \end{itemize}

  \item IP Addresssing in Subnets: CIDR

    \begin{itemize}

      \item CIDR: Classless Inter Domain Routing (pronouned ``cider'')

        \begin{itemize}

          \item Network portion (aka prefix) of address of arbitrary length

          \item Address format (by convention): \textsc{a.b.c.d.x}, where \textsc{x} os the number of bits in the network portion of the address

        \end{itemize}

      \item Network address (subnet address): network portion and 0s in the host portion/\textsc{x}

      \item Subnet mask: binary mask of 1s in teh subnet portion and 0s in the host portion $\to$ \textsc{x}

        \begin{itemize}

          \item The subnet mask can be ANDed with an IP address to obtain the network address

        \end{itemize}

      \item Recipe for identifying subnets

        \begin{itemize}

          \item Detach each interface from its host or router, creating ``islands'' of isolated networks

          \item Each isolated network is a subnet

        \end{itemize}

    \end{itemize}

  \item Longest Prefix Matching

    \begin{itemize}

      \item When looking for forwarding table entry for a given destination address, use longest address prefix that matches destination address.

    \end{itemize}

  \item Forwarding in Access Networks

    \begin{itemize}

      \item Forwarding tables in routers of an access network have an entry for their subnets

      \item When a datagram reaches a router in an access network, it looks at the destination address of the datagram, and checks which subnet inside the network it belongs to. How?

        \begin{itemize}

          \item AND the destination address with the mask for each subnet entry in the table

          \item Check to see if the result is the prefix in the entry

        \end{itemize}

    \end{itemize}

  \item Forwarding in the Network Core

    \begin{itemize}

      \item Routers in ISPs and backbones in the middle of the internet must know which way to go to get to every network and no simple default will work

        \begin{itemize}

          \item This can make for a very large table

            \begin{itemize}

              \item Routers must perform a lookup in this table for every datagram they forward

            \end{itemize}

        \end{itemize}

    \end{itemize}

  \item Hierarchical Addressing: Route Aggregation

    \begin{itemize}

      \item Hierarchical addressing allows efficient advertisement for routing information

    \end{itemize}

  \item How Are IP Addresses Assigned?

    \begin{itemize}

      \item Hard-coded by system administrator $\to$ fixed IP address

      \item DHCP: Dynamic Host Configuration Protocol

        \begin{itemize}

          \item Can renew its lease on address in use

          \item Allows reuse of addresses (only hold address while connected/on)

          \item Support for mobile users who join/leave network

        \end{itemize}

    \end{itemize}

  \item DHCP: More than IP Addresses

    \begin{itemize}

      \item DHCP can return more than just allocated IP addresses on a subnet:

        \begin{itemize}

          \item Address of first-hop router for client

          \item Name and IP address of local DNS server

          \item Subnet mask (indicating network versus host portion of address)

        \end{itemize}

    \end{itemize}

  \item NAT: Network Address Translation

    \begin{itemize}

      \item All devices in local network have 32-bit IP addresses in a ``private'' IP address space that can only be used in local networks

      \item ``Private'' IP address space corresponds to prefixes: 10.0.0.0/8, 172.16.0.0/12 and 192.168.0.0/16

        \begin{itemize}

          \item Defined by IANA (Internet Assigned Numbers Authority) $\to$ department of ICANN

        \end{itemize}

      \item Advantages:

        \begin{itemize}

          \item Private IP addresses can be reused in different private networks

          \item Just one IP address needed from provider ISP for all devices

        \end{itemize}

      \item Implementation: NAT router (or NAT box)

      \item Outgoing datagrams: replace (source IP address, port \#) of every outgoing datagram to (NAT IP address, new port \#)

        \begin{itemize}

          \item Remote clients/servers will respond using (NAT IP address, new port \#) as destination address

        \end{itemize}

      \item Store in NAT translation table every (source IP address, port \#) to (NAT IP address, new port) translation pair

      \item Incoming datagrams: replace (NAT IP address, new port \#) in destination fields of every incoming datagram with corresponding (source IP address, port \#) stored in NAT table

      \item NAT has been controversial:

        \begin{itemize}

          \item Routers ``should'' only process up to layer 3

          \item Address ``shortage'' should be solved by IPv6

          \item Violates end-to-end argument (port number manipulation by network-layer device)

          \item NAT traversal: what if client wants to connect to server behind NAT?

        \end{itemize}

      \item But NAT is here to stay:

        \begin{itemize}

          \item Extensively used in home and institutional networks, 4G/5G cellular networks

        \end{itemize}

    \end{itemize}

  \item IPv6: Motivation

    \begin{itemize}

      \item Initial motivation: 32-bit IPv4 address space would be completely allocated

      \item Additional motivation:

        \begin{itemize}

          \item Speed processing/forwarding: 40-byte fixed length header

            \begin{itemize}

              \item Extension headers: optional headers can be added after the fixed IPv6 header

            \end{itemize}

          \item Enable different network-layer treatment of ``flows''

        \end{itemize}

    \end{itemize}

  \item Transition from IPv4 to IPv6

    \begin{itemize}

      \item Not all routers can be upgraded simultaneously

        \begin{itemize}

          \item No ``flag delays''

          \item How will network operate with mixed IPv4 and IPv6 routers?

        \end{itemize}

      \item Tunneling: Packet within a packet

        \begin{itemize}

          \item IPv6 datagram carried as payload in IPv4 datagram among IPv4 routers (``packet within a packet'')

          \item TUnneling used extensively in other contexts (4G/5G)

        \end{itemize}

    \end{itemize}

  \item Flow Table Abstraction

    \begin{itemize}

      \item Flow: defined by header fields

      \item Generalized Forwarding

        \begin{itemize}

          \item Match: pattern values in packet header fields

          \item Actions: for matched packet: drop, forward, modify matched packet or send matched packet to controller

          \item Priority: disambiguate overlapping packets

          \item Coutners: \# bytes and packets

        \end{itemize}

    \end{itemize}

  \item OpenFlow Abstraction

    \begin{itemize}

      \item Match and Action: abstraction unifies different kinds of devices

      \item Router:

        \begin{itemize}

          \item Match: longest destination IP prefix

          \item Action: forward out a link

        \end{itemize}

      \item Firewall:

        \begin{itemize}

          \item Match: IP addresses and TCP/UDP port numbers

          \item Action: permit of deny

        \end{itemize}

      \item Swtich

        \begin{itemize}

          \item Match: destination MAC adress (link layer address)

          \item Action: Forward or flood

        \end{itemize}

      \item NAT

        \begin{itemize}

          \item Match: IP address and port

          \item Action: rewrite address and port

        \end{itemize}

    \end{itemize}

  \item Middlebox

    \begin{itemize}

      \item Any intermediary box performing functions apart from normal, standard functions of an IP router

      \item Initially: proprietary (closed) hardware solutions

      \item Move towards ``whitebox'' hardware implementing open API

        \begin{itemize}

          \item Move away from proprietary hardware solutions

          \item Programmable local actions via match and action

          \item Move towards innovation/differentiation in software

        \end{itemize}

      \item SDN: (logically) centralized control and configuration management often in private/public cloud

        \begin{itemize}

          \item Network Function Virtualization (NFV): programmable services over white box networking, computation, and storage

            \begin{itemize}

              \item Allows for programmable network devices

            \end{itemize}

        \end{itemize}

    \end{itemize}

\end{itemize}

\end{document}

