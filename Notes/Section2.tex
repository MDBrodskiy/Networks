%%%%%%%%%%%%%%%%%%%%%%%%%%%%%%%%%%%%%%%%%%%%%%%%%%%%%%%%%%%%%%%%%%%%%%%%%%%%%%%%%%%%%%%%%%%%%%%%%%%%%%%%%%%%%%%%%%%%%%%%%%%%%%%%%%%%%%%%%%%%%%%%%%%%%%%%%%%%%%%%%%%
% Written By Michael Brodskiy
% Class: Fundamentals of Networks
% Professor: E. Bernal Mor
%%%%%%%%%%%%%%%%%%%%%%%%%%%%%%%%%%%%%%%%%%%%%%%%%%%%%%%%%%%%%%%%%%%%%%%%%%%%%%%%%%%%%%%%%%%%%%%%%%%%%%%%%%%%%%%%%%%%%%%%%%%%%%%%%%%%%%%%%%%%%%%%%%%%%%%%%%%%%%%%%%%

\documentclass[12pt]{article} 
\usepackage{alphalph}
\usepackage[utf8]{inputenc}
\usepackage[russian,english]{babel}
\usepackage{titling}
\usepackage{amsmath}
\usepackage{graphicx}
\usepackage{enumitem}
\usepackage{amssymb}
\usepackage[super]{nth}
\usepackage{everysel}
\usepackage{ragged2e}
\usepackage{geometry}
\usepackage{multicol}
\usepackage{fancyhdr}
\usepackage{cancel}
\usepackage{siunitx}
\usepackage{physics}
\usepackage{tikz}
\usepackage{mathdots}
\usepackage{yhmath}
\usepackage{cancel}
\usepackage{color}
\usepackage{array}
\usepackage{multirow}
\usepackage{gensymb}
\usepackage{tabularx}
\usepackage{extarrows}
\usepackage{booktabs}
\usepackage{lastpage}
\usepackage{float}
\usetikzlibrary{fadings}
\usetikzlibrary{patterns}
\usetikzlibrary{shadows.blur}
\usetikzlibrary{shapes}

\geometry{top=1.0in,bottom=1.0in,left=1.0in,right=1.0in}
\newcommand{\subtitle}[1]{%
  \posttitle{%
    \par\end{center}
    \begin{center}\large#1\end{center}
    \vskip0.5em}%

}
\usepackage{hyperref}
\hypersetup{
colorlinks=true,
linkcolor=blue,
filecolor=magenta,      
urlcolor=blue,
citecolor=blue,
}


\title{The Application Layer}
\date{\today}
\author{Michael Brodskiy\\ \small Professor: E. Bernal Mor}

\begin{document}

\maketitle

\begin{itemize}

  \item Network Applications

    \begin{itemize}

      \item Social networking

      \item Web

      \item Text messaging

      \item E-mail

      \item Multi-user network games

      \item Streaming stored video (YouTube, Hulu, Netflix)

      \item P2P File Sharing

      \item And many more

    \end{itemize}

  \item Creating Network Applications

    \begin{itemize}

      \item Write programs that:

        \begin{itemize}

          \item Run on (different) end systems

          \item Communicate over network

          \item For example, web server software communicates with browser software

        \end{itemize}

      \item No need to write software for network-core devices (intermediate nodes)

        \begin{itemize}

          \item Network-core devices do not run user applications

          \item Applications on end systems allow for rapid application development and propagation

        \end{itemize}

    \end{itemize}

  \item Application Architecture

    \begin{itemize}

      \item Network architecture — a set of layers and protocols

        \begin{itemize}

          \item It is fixed, and provides the network application developer with specific set of services

        \end{itemize}

      \item Application Architecture — define how the application is structured over various end systems

        \begin{itemize}

          \item Designed by the application developer

          \item Predominant architectural paradigms

            \begin{itemize}

              \item Client-server

              \item Peer-to-peer (P2P)

            \end{itemize}

        \end{itemize}

    \end{itemize}

  \item Client-server Architecture

    \begin{itemize}
        
      \item Server

        \begin{itemize}

          \item Always-on host

          \item Permanent IP-address (like ID)

          \item Often in data centers, for scaling

        \end{itemize}

      \item Clients

        \begin{itemize}

          \item Contact, communicate with server

          \item May be intermittently connected

          \item May have dynamic IP addresses

          \item Do not communicate directly with each other

        \end{itemize}

      \item Examples: HTTP, IMAP, SFTP

    \end{itemize}

  \item Peer-Peer (P2P) Architecture

    \begin{itemize}

      \item No always-on server

      \item Arbitrary end systems directly communicate

      \item Peers request service from other peers, provide service in return to other peers

        \begin{itemize}

          \item Self scalability — new peers bring new service capacity, as well as new service demands

        \end{itemize}

      \item Peers are intermittently connected and change IP addresses 

        \begin{itemize}

          \item Complex management

        \end{itemize}

      \item Example: P2P File Sharing

    \end{itemize}

  \item Process Communication

    \begin{itemize}

      \item Process — program running within a host

        \begin{itemize}

          \item Within same host, two processes communicate using inter-process communication, defined by OS (Operating System)

          \item Processes in different hosts communicate by exchansing messages

        \end{itemize}

      \item Client process — process that initiates communication

      \item Server process — process that wants to be contacted

      \item Note: applications with P2P architectures have client processes 7& server processes

    \end{itemize}

  \item Sockets

    \begin{itemize}

      \item Process send/receives messages to/from its socket

      \item Socket analogous to door

        \begin{itemize}

          \item Sending process shoves message out the door

          \item Sending process relies on transport infrastructure on other side of door to deliver message to socket at receiving process

        \end{itemize}

    \end{itemize}

  \item Addressing Processes

    \begin{itemize}

      \item To receive messages, a process must have an identifier

      \item Host device has a unique IP address

      \item Identifier includes both IP address and port numbers associated with process on host

        \begin{itemize}

          \item HTTP server: 80

          \item Mail server: 25

        \end{itemize}

      \item To send HTTP message to gaia.cs.umass.edu web server:

        \begin{itemize}

          \item IP address: 128.119.245.12

          \item Port number: 80

        \end{itemize}

    \end{itemize}

  \item An Application Layer Protocol Defines:

    \begin{itemize}

      \item Types of messages exchanged

        \begin{itemize}

          \item Example: request, response

        \end{itemize}

      \item Message syntax

        \begin{itemize}

          \item What fields in messages \& how fields are delineated

        \end{itemize}

      \item Message semantics

        \begin{itemize}

          \item Meaning of information in fields

        \end{itemize}

      \item Rules for when and how processes send \& respond to messages

    \end{itemize}

  \item Application Layer Protocols can be

    \begin{itemize}

      \item Open protocols

        \begin{itemize}

          \item Defined in RFCs, everyone has access to protocol definition

          \item Allows for interoperability

          \item Example: HTTP, SMTP

        \end{itemize}

      \item Proprietary protocols

        \begin{itemize}

          \item Example: Skype

        \end{itemize}

    \end{itemize}

  \item Transport Layer Services for Applications

    \begin{itemize}

      \item Transport layer is on the other side of the ``door''

      \item There are multiple Transport-layer protocols that provide different services

      \item The application developer must choose a Transport-layer protocol, depending on the services needed by the application

        \begin{itemize}

          \item Examples: priority mail, express mail, certified mail

        \end{itemize}

      \item A Transport-layer protocol can provide a different array of services

    \end{itemize}

  \item Transport Services

    \begin{itemize}

      \item Data integrity/reliable transport

        \begin{itemize}

          \item Some apps (\textit{e}.\textit{g}.\ file transfer, web transactions) require 100\% reliable data transfer

          \item Other apps (\textit{e}.\textit{g}.\ audio) can tolerate some loss

        \end{itemize}

      \item Timing

        \begin{itemize}

          \item Some apps (\textit{e}.\textit{g}. Internet telephony, interactive games) require low delay to be ``effective''

        \end{itemize}

      \item Throughput

        \begin{itemize}

          \item Some apps (e.g., multimedia) require minimum amount of throughput to be ``effective''

          \item Other apps (``elastic apps'') make use of whatever throughput they get

        \end{itemize}

      \item Security

        \begin{itemize}

          \item Encryption, data integrity, \ldots

        \end{itemize}

    \end{itemize}

  \item Internet Transport Protocol Services

    \begin{itemize}

      \item TCP Service

        \begin{itemize}

          \item Reliable transport between sending and receiving processes

          \item Flow control — sender will not overwhelm receiver

          \item Congestion control — throttle sender when network overloaded

          \item Does not provide timing, minimum throughput guarantee, security

          \item Connection-oriented service: setup required between client and service processes

        \end{itemize}

      \item UDP Service:

        \begin{itemize}

          \item Unreliable data transfer between sending and receiving process

          \item Does not provide reliability, flow control, congestion control, timing, throughput guarantee, security, or connection setup

          \item Connectionless service: no setup required

        \end{itemize}

    \end{itemize}

  \item Vanilla TCP \& UDP sockets

    \begin{itemize}

      \item No encryption

      \item Clear text passwords sent into socket traverse Internet in clear text

    \end{itemize}

  \item Transport Layer Security (TLS)

    \begin{itemize}

      \item Provides encrypted TCP connections

      \item Data integrity

      \item End-point authentication

      \item TSL implemented in Application Layer

        \begin{itemize}

          \item Applications use TLS libraries, that use TCP in turn

        \end{itemize}

      \item TLS socket API

        \begin{itemize}

          \item Clear text sent into socket traverse Internet encrypted

        \end{itemize}

    \end{itemize}

  \item Designing Network Applications

    \begin{itemize}

      \item It is a complex process

      \item Requires knowledge of programming, software engineering, and networking

      \item From a networking point of view, there are two major decisions:

        \begin{enumerate}

          \item Type of application (aka Application Architecture)

            \begin{itemize}

              \item Client-server vs.\ peer-to-peer

            \end{itemize}

          \item Services requested to the Transport Layer

            \begin{itemize}

              \item \textit{E}.\textit{g}.\  reliable vs. unreliable data transfer

            \end{itemize}

        \end{enumerate}

    \end{itemize}

\end{itemize}

\end{document}

