%%%%%%%%%%%%%%%%%%%%%%%%%%%%%%%%%%%%%%%%%%%%%%%%%%%%%%%%%%%%%%%%%%%%%%%%%%%%%%%%%%%%%%%%%%%%%%%%%%%%%%%%%%%%%%%%%%%%%%%%%%%%%%%%%%%%%%%%%%%%%%%%%%%%%%%%%%%%%%%%%%%
% Written By Michael Brodskiy
% Class: Fundamentals of Networks
% Professor: E. Bernal Mor
%%%%%%%%%%%%%%%%%%%%%%%%%%%%%%%%%%%%%%%%%%%%%%%%%%%%%%%%%%%%%%%%%%%%%%%%%%%%%%%%%%%%%%%%%%%%%%%%%%%%%%%%%%%%%%%%%%%%%%%%%%%%%%%%%%%%%%%%%%%%%%%%%%%%%%%%%%%%%%%%%%%

\documentclass[12pt]{article} 
\usepackage{alphalph}
\usepackage[utf8]{inputenc}
\usepackage[russian,english]{babel}
\usepackage{titling}
\usepackage{amsmath}
\usepackage{graphicx}
\usepackage{enumitem}
\usepackage{amssymb}
\usepackage[super]{nth}
\usepackage{everysel}
\usepackage{ragged2e}
\usepackage{geometry}
\usepackage{multicol}
\usepackage{fancyhdr}
\usepackage{cancel}
\usepackage{siunitx}
\usepackage{physics}
\usepackage{tikz}
\usepackage{mathdots}
\usepackage{yhmath}
\usepackage{cancel}
\usepackage{color}
\usepackage{array}
\usepackage{multirow}
\usepackage{gensymb}
\usepackage{tabularx}
\usepackage{extarrows}
\usepackage{booktabs}
\usepackage{lastpage}
\usepackage{float}
\usetikzlibrary{fadings}
\usetikzlibrary{patterns}
\usetikzlibrary{shadows.blur}
\usetikzlibrary{shapes}

\geometry{top=1.0in,bottom=1.0in,left=1.0in,right=1.0in}
\newcommand{\subtitle}[1]{%
  \posttitle{%
    \par\end{center}
    \begin{center}\large#1\end{center}
    \vskip0.5em}%

}
\usepackage{hyperref}
\hypersetup{
colorlinks=true,
linkcolor=blue,
filecolor=magenta,      
urlcolor=blue,
citecolor=blue,
}


\title{The Application Layer}
\date{\today}
\author{Michael Brodskiy\\ \small Professor: E. Bernal Mor}

\begin{document}

\maketitle

\begin{itemize}

  \item Network Applications

    \begin{itemize}

      \item Social networking

      \item Web

      \item Text messaging

      \item E-mail

      \item Multi-user network games

      \item Streaming stored video (YouTube, Hulu, Netflix)

      \item P2P File Sharing

      \item And many more

    \end{itemize}

  \item Creating Network Applications

    \begin{itemize}

      \item Write programs that:

        \begin{itemize}

          \item Run on (different) end systems

          \item Communicate over network

          \item For example, web server software communicates with browser software

        \end{itemize}

      \item No need to write software for network-core devices (intermediate nodes)

        \begin{itemize}

          \item Network-core devices do not run user applications

          \item Applications on end systems allow for rapid application development and propagation

        \end{itemize}

    \end{itemize}

  \item Application Architecture

    \begin{itemize}

      \item Network architecture — a set of layers and protocols

        \begin{itemize}

          \item It is fixed, and provides the network application developer with specific set of services

        \end{itemize}

      \item Application Architecture — define how the application is structured over various end systems

        \begin{itemize}

          \item Designed by the application developer

          \item Predominant architectural paradigms

            \begin{itemize}

              \item Client-server

              \item Peer-to-peer (P2P)

            \end{itemize}

        \end{itemize}

    \end{itemize}

  \item Client-server Architecture

    \begin{itemize}
        
      \item Server

        \begin{itemize}

          \item Always-on host

          \item Permanent IP-address (like ID)

          \item Often in data centers, for scaling

        \end{itemize}

      \item Clients

        \begin{itemize}

          \item Contact, communicate with server

          \item May be intermittently connected

          \item May have dynamic IP addresses

          \item Do not communicate directly with each other

        \end{itemize}

      \item Examples: HTTP, IMAP, SFTP

    \end{itemize}

  \item Peer-Peer (P2P) Architecture

    \begin{itemize}

      \item No always-on server

      \item Arbitrary end systems directly communicate

      \item Peers request service from other peers, provide service in return to other peers

        \begin{itemize}

          \item Self scalability — new peers bring new service capacity, as well as new service demands

        \end{itemize}

      \item Peers are intermittently connected and change IP addresses 

        \begin{itemize}

          \item Complex management

        \end{itemize}

      \item Example: P2P File Sharing

    \end{itemize}

  \item Process Communication

    \begin{itemize}

      \item Process — program running within a host

        \begin{itemize}

          \item Within same host, two processes communicate using inter-process communication, defined by OS (Operating System)

          \item Processes in different hosts communicate by exchansing messages

        \end{itemize}

      \item Client process — process that initiates communication

      \item Server process — process that wants to be contacted

      \item Note: applications with P2P architectures have client processes 7& server processes

    \end{itemize}

  \item Sockets

    \begin{itemize}

      \item Process send/receives messages to/from its socket

      \item Socket analogous to door

        \begin{itemize}

          \item Sending process shoves message out the door

          \item Sending process relies on transport infrastructure on other side of door to deliver message to socket at receiving process

        \end{itemize}

    \end{itemize}

  \item Addressing Processes

    \begin{itemize}

      \item To receive messages, a process must have an identifier

      \item Host device has a unique IP address

      \item Identifier includes both IP address and port numbers associated with process on host

        \begin{itemize}

          \item HTTP server: 80

          \item Mail server: 25

        \end{itemize}

      \item To send HTTP message to gaia.cs.umass.edu web server:

        \begin{itemize}

          \item IP address: 128.119.245.12

          \item Port number: 80

        \end{itemize}

    \end{itemize}

  \item An Application Layer Protocol Defines:

    \begin{itemize}

      \item Types of messages exchanged

        \begin{itemize}

          \item Example: request, response

        \end{itemize}

      \item Message syntax

        \begin{itemize}

          \item What fields in messages \& how fields are delineated

        \end{itemize}

      \item Message semantics

        \begin{itemize}

          \item Meaning of information in fields

        \end{itemize}

      \item Rules for when and how processes send \& respond to messages

    \end{itemize}

  \item Application Layer Protocols can be

    \begin{itemize}

      \item Open protocols

        \begin{itemize}

          \item Defined in RFCs, everyone has access to protocol definition

          \item Allows for interoperability

          \item Example: HTTP, SMTP

        \end{itemize}

      \item Proprietary protocols

        \begin{itemize}

          \item Example: Skype

        \end{itemize}

    \end{itemize}

  \item Transport Layer Services for Applications

    \begin{itemize}

      \item Transport layer is on the other side of the ``door''

      \item There are multiple Transport-layer protocols that provide different services

      \item The application developer must choose a Transport-layer protocol, depending on the services needed by the application

        \begin{itemize}

          \item Examples: priority mail, express mail, certified mail

        \end{itemize}

      \item A Transport-layer protocol can provide a different array of services

    \end{itemize}

  \item Transport Services

    \begin{itemize}

      \item Data integrity/reliable transport

        \begin{itemize}

          \item Some apps (\textit{e}.\textit{g}.\ file transfer, web transactions) require 100\% reliable data transfer

          \item Other apps (\textit{e}.\textit{g}.\ audio) can tolerate some loss

        \end{itemize}

      \item Timing

        \begin{itemize}

          \item Some apps (\textit{e}.\textit{g}. Internet telephony, interactive games) require low delay to be ``effective''

        \end{itemize}

      \item Throughput

        \begin{itemize}

          \item Some apps (e.g., multimedia) require minimum amount of throughput to be ``effective''

          \item Other apps (``elastic apps'') make use of whatever throughput they get

        \end{itemize}

      \item Security

        \begin{itemize}

          \item Encryption, data integrity, \ldots

        \end{itemize}

    \end{itemize}

  \item Internet Transport Protocol Services

    \begin{itemize}

      \item TCP Service

        \begin{itemize}

          \item Reliable transport between sending and receiving processes

          \item Flow control — sender will not overwhelm receiver

          \item Congestion control — throttle sender when network overloaded

          \item Does not provide timing, minimum throughput guarantee, security

          \item Connection-oriented service: setup required between client and service processes

        \end{itemize}

      \item UDP Service:

        \begin{itemize}

          \item Unreliable data transfer between sending and receiving process

          \item Does not provide reliability, flow control, congestion control, timing, throughput guarantee, security, or connection setup

          \item Connectionless service: no setup required

        \end{itemize}

    \end{itemize}

  \item Vanilla TCP \& UDP sockets

    \begin{itemize}

      \item No encryption

      \item Clear text passwords sent into socket traverse Internet in clear text

    \end{itemize}

  \item Transport Layer Security (TLS)

    \begin{itemize}

      \item Provides encrypted TCP connections

      \item Data integrity

      \item End-point authentication

      \item TSL implemented in Application Layer

        \begin{itemize}

          \item Applications use TLS libraries, that use TCP in turn

        \end{itemize}

      \item TLS socket API

        \begin{itemize}

          \item Clear text sent into socket traverse Internet encrypted

        \end{itemize}

      \item Datagram Transport Layer Service (DTLS) protocol

        \begin{itemize}

          \item Adaptation of TLS to run over connectionless protocols such as UDP

        \end{itemize}

    \end{itemize}

  \item Designing Network Applications

    \begin{itemize}

      \item It is a complex process

      \item Requires knowledge of programming, software engineering, and networking

      \item From a networking point of view, there are two major decisions:

        \begin{enumerate}

          \item Type of application (aka Application Architecture)

            \begin{itemize}

              \item Client-server vs.\ peer-to-peer

            \end{itemize}

          \item Services requested to the Transport Layer

            \begin{itemize}

              \item \textit{E}.\textit{g}.\  reliable vs. unreliable data transfer

            \end{itemize}

        \end{enumerate}

    \end{itemize}

  \item Web and HTTP

    \begin{itemize}

      \item Web page consists of several objects, each of which can be store on different Web servers

      \item Object can be HTML, JPEG, Java applet, audio file, etc.

      \item Web page consists of a base HTML-file, which includes several referenced objects

      \item Each object is addressable by a URL (Uniform Resource Locator), \textit{e}.\textit{g}.,

        \begin{center}
          $\underbrace{\text{\texttt{www.someschool.edu/}}}_{\text{host name}}\underbrace{\text{\texttt{someDept/pic.gif}}}_{\text{path name}}$
        \end{center}

    \end{itemize}

  \item HTTP Overview

    \begin{itemize}

      \item HTTP — Hypertext Transfer Protcol

      \item Web's application layer protocol

      \item Client/server model

        \begin{itemize}

          \item Client: Browser that requests, receives (using HTTP protocol) and ``displays'' Web objects

          \item Server: Web server sends (using HTTP protocol) objects in response to requests

        \end{itemize}

      \item Versions

        \begin{itemize}

          \item HTTP/1.0 (RFC1945)

            \begin{itemize}

              \item Original HTTP version (early 1990s)

            \end{itemize}

          \item HTTP/1.1 (RFC7230,\ldots)

            \begin{itemize}

              \item Used by most of the HTTP transactions

            \end{itemize}

          \item HTTP/2 (RFC7540,\ldots)

            \begin{itemize}

              \item Standardized in 2015 and increasingly used by browsers and servers

            \end{itemize}

          \item HTTP/3 (RFC9114)

            \begin{itemize}

              \item IETF published it as a proposed standard in June 2022

            \end{itemize}

        \end{itemize}

      \item HTTP used TCP (except HTTP/3):

        \begin{itemize}

          \item Client initiates TCP connect (creates socket) to server, port 80

          \item Server accepts TCP connection from client

          \item HTTP messages (application-layer protocol messages) exchanged between browser (HTTP client) and Web server (HTTP server)

          \item TCP connection closed

        \end{itemize}

      \item HTTP is ``stateless''

        \begin{itemize}

          \item Server maintains no information about past client requests

        \end{itemize}

      \item Protocols that maintain ``state'' are complex

        \begin{itemize}

          \item Past history (state) must be maintained

          \item If server/client crashes, their view of ``state'' may be inconsistent, must be reconciled

        \end{itemize}

      \item Non-persistent HTTP

        \begin{enumerate}

          \item TCP connection opened

          \item At most one object sent over TCP connection

          \item TCP connection closed

        \end{enumerate}

        \begin{itemize}

          \item Downloading multiple objects requires multiple connections

        \end{itemize}

      \item Persistent HTTP

        \begin{enumerate}

          \item TCP connection opened to a server

          \item Multiple objects can be sent over single TCP connection between client, and that server

          \item TCP connection closed

        \end{enumerate}

      \item RTT (Round-Trip Time) definition: time for a small packet to travel from client to server and back

        \begin{itemize}

          \item Includes propagation, queueing, and processing delays

        \end{itemize}

      \item Non-persistent HTTP response time (per object)

        \begin{itemize}

          \item One RTT to initiate TCP connection

          \item One RTT for HTTP request/response

          \item File/object transmission time

          \item Non-persistent HTTP time $\approx$ 2RTT + transmission time (per object)

        \end{itemize}

      \item Non-persistent HTTP issues:

        \begin{itemize}

          \item Requires 2 RTTs per object

          \item OS overhead for each TCP connection

          \item Browsers often open parallel TCP connections to fetch referenced objects in parallel

        \end{itemize}

      \item Persistent HTTP/1.1:

        \begin{itemize}

          \item Server leaves connection open after sending response

          \item Subsequent HTTP messages between same client/server sent over open connection

          \item Client sends requests as soon as it encounters a referenced object

          \item As little as one RTT for all the referenced objects (cutting response time in half)

        \end{itemize}

      \item HTTP Request Message

        \begin{itemize}

          \item Two types of HTTP messages: request, response

          \item HTTP request message is in ASCII (human-readable format)

        \end{itemize}

    \end{itemize}

  \item HTTP Methods

    \begin{itemize}

      \item GET method

        \begin{itemize}

          \item To request an object

          \item If user data (like form input), the data is included in URL field of HTTP GET request message (following a '?'):

            \begin{center}
              \texttt{ www.somesite.com/animalsearch?table\&chair}
            \end{center}

        \end{itemize}

      \item POST method

        \begin{itemize}

          \item Web page often includes form input

          \item User input sent from client to server in entity body of HTTP POST request message

        \end{itemize}

      \item HEAD method

        \begin{itemize}

          \item Requests headers (only) that would be returned if specified URL were requested  with an HTTP GET method

        \end{itemize}

      \item PUT method

        \begin{itemize}

          \item Uploads new file (object) to server

          \item Completely replaces file that exists at specified URL with content in entity body

        \end{itemize}

      \item DELETE method

        \begin{itemize}

          \item Deletes file specified in the URL field

        \end{itemize}

    \end{itemize}

  \item HTTP Response Status Codes

    \begin{itemize}

      \item 200 — OK

        \begin{itemize}

          \item Request succeeded, requested object later in this message

        \end{itemize}

      \item 301 — Moved Permanently

        \begin{itemize}

          \item Requested object moved, new location specified later in this msg (Location:)

        \end{itemize}

      \item 400 — Bad Request

        \begin{itemize}

          \item Request msg not understood by server

        \end{itemize}

      \item 404 — Not Found

        \begin{itemize}

          \item Requested document not found on this server

        \end{itemize}

      \item 505 HTTP — Version Not Supported

    \end{itemize}

  \item HTTP/2

    \begin{itemize}

      \item Key goal: decreased delay in multiple-object HTTP requests

      \item HTTP/1.1: introduced multiple, pipelined GETs over single TCP connection (persistent HTTP)

        \begin{itemize}

          \item FCFS (First-Come-First-Served) scheduling: server responds in-order to GET requests

          \item Head-Of-Line (HOL) blocking: with FCFS, small object may have to wait for transmission behind large object(s)

          \item Loss recovery (retransmitting lost TCP segments) stalls object transmission

        \end{itemize}

      \item HTTP/2: increased flexibility at server in sending objects to client

        \begin{itemize}

          \item Methods, status codes, most header fields unchanged from HTTP 1.1

          \item Transmission order of requested objects based on client-specified object priority (not necessarily FCFS)

          \item Push unrequested objects to client

          \item Divide objects into frames, schedule frames to mitigate HOL blocking

        \end{itemize}

    \end{itemize}

  \item HTTP/2 to HTTP/3

    \begin{itemize}

      \item HTTP/2 over single TCP connection means:

        \begin{itemize}

          \item Recovery from packet loss still stalls all object transmissions
            
            \begin{itemize}

              \item As in HTTP 1.1, browsers have incentive to open multiple parallel TCP connections to reduce stalling and increase overall application data throughput

            \end{itemize}

          \item No security over vanilla TCP connection

        \end{itemize}

      \item HTTP/3 operates over QUIC, a transport protocol built on UDP

        \begin{itemize}

          \item Adds security, per object error, and congestion-control (more pipelining) over UDP

        \end{itemize}

    \end{itemize}

  \item Maintaining User/Server State: Cookies

    \begin{itemize}

      \item Recall: HTTP GET/response interaction is stateless

      \item Web sites and client browser use cookies to maintain some state between transactions

      \item Four components:

        \begin{enumerate}

          \item Cookie header line of HTTP response message

          \item Cookie header line in next HTTP request message

          \item Cookie file kept on user's host, managed by user's browser

          \item Back-end database at Web site

        \end{enumerate}

    \end{itemize}

\end{itemize}

\end{document}

