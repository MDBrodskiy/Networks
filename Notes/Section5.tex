%%%%%%%%%%%%%%%%%%%%%%%%%%%%%%%%%%%%%%%%%%%%%%%%%%%%%%%%%%%%%%%%%%%%%%%%%%%%%%%%%%%%%%%%%%%%%%%%%%%%%%%%%%%%%%%%%%%%%%%%%%%%%%%%%%%%%%%%%%%%%%%%%%%%%%%%%%%%%%%%%%%
% Written By Michael Brodskiy
% Class: Fundamentals of Networks
% Professor: E. Bernal Mor
%%%%%%%%%%%%%%%%%%%%%%%%%%%%%%%%%%%%%%%%%%%%%%%%%%%%%%%%%%%%%%%%%%%%%%%%%%%%%%%%%%%%%%%%%%%%%%%%%%%%%%%%%%%%%%%%%%%%%%%%%%%%%%%%%%%%%%%%%%%%%%%%%%%%%%%%%%%%%%%%%%%

\include{Includes.tex}

\title{The Network Layer: Control Plane}
\date{\today}
\author{Michael Brodskiy\\ \small Professor: E. Bernal Mor}

\begin{document}

\maketitle

\begin{itemize}

  \item Network-Layer Functions

    \begin{itemize}

      \item Forwarding (data plane)

      \item Routing: determine route taken by packets from source to destination (control plane)

        \begin{itemize}

          \item Two approaches to structuring a network control plane:

            \begin{itemize}

              \item Per-router plane (traditional)

              \item Software-defined

            \end{itemize}

        \end{itemize}

    \end{itemize}

  \item Per-Router Control Plane

    \begin{itemize}

      \item Individual routing algorithm components in each and every router interact in the control plane

    \end{itemize}

  \item Logically Centralized Control Plane (SDN)

    \begin{itemize}

      \item Remote controller computers, installs forwarding tables (aka flow tables) in routers)

    \end{itemize}

  \item Routing Protocols

    \begin{itemize}

      \item Routing protocol goal: determine ``good'' paths (equivalently, routes) from sending hosts to receiving hosts, through network of routers

        \begin{itemize}

          \item Path: sequence of routers that packets traverse from given initial source host to destination host

          \item ``Good'': least ``cost'', ``fastest'', ``least congested''

          \item Routing is a top networking challenge

        \end{itemize}

    \end{itemize}

  \item Graph Abstraction: Link Costs

    \begin{itemize}

      \item $c_{a,b}$ is the cost of a direct link connecting $a$ and $b$

        \begin{itemize}

          \item Cost is defined by network operator: could always be 1, or inversely related to link capacity, or proportional to length, etc.

        \end{itemize}

      \item The overall cost is a sum of all the costs from link to link

      \item The goal of a routing algorithm is to identify the least-cost path (aka shortest path) from sources to destination

      \item If all links have the same cost, the least-cost path is the path with the minimal number of links

    \end{itemize}

  \item Routing Algorithm Classification

    \begin{itemize}

      \item Centralized or global: all routers have complete topology, link cost info (``link state'' algorithms)

      \item Decentralized: iterative process of computation, exchange of info with neighbors (``distance vector'' algorithms)

      \item Static: routes change slowly over time

      \item Dynamic: routes change more quickly (periodic updates or in response to link cost changes)

    \end{itemize}

  \item Djikstra's Link-State Routing Algorithm

    \begin{itemize}

      \item Centralized: network topology and link costs known to all nodes

        \begin{itemize}

          \item Accomplished vie ``link state broadcast''

          \item All nodes have same info

        \end{itemize}

      \item Computes least cost paths from one node (``source'') to all other nodes

        \begin{itemize}

          \item Gives forwarding table for that node

        \end{itemize}

      \item Iterative: after $k$ iterations, know least cost path to $k$ destinations

      \item Notation:

        \begin{itemize}

          \item $c_{x,y}$: direct link cost from node $x$ to $y$; $c_{x,y}=\infty$ if not direct neighbors
            
          \item $D(v)$: current estimate of cost of least-cost-path from source to destination $v$

          \item $p(v)$: predecessor node along path from source to $v$

          \item $N'$: set of nodes whose least-cost-path is definitively known

          \item Ties can exist, and are broken arbitrarily

          \item Construct least-cost-path tree by tracing predecessor nodes

        \end{itemize}

      \item Djikstra's Algorithm Complexity: $n$ nodes

        \begin{itemize}

          \item Each of $n$ iterations: need to check all the nodes, $w$, not in $N'$

          \item $n(n+1)/2$ comparisons: $O(n^2)$ complexity

          \item More efficient implementations possible $O(n\log(n))$

        \end{itemize}

      \item Oscillations possible: when link costs depend on traffic volume

    \end{itemize}

  \item Distance Vector Algorithm

    \begin{itemize}

      \item Based on Bellman-Ford (BF) Equation [dynamic programming]

      \item Let $d_x(y)$: cost of least-cost path from $x$ to $y$, then:

        $$d_x(y)=\text{min}_v(c_{x,v}+d_v(y))$$

      \item $D_x(y)$ is the estimate of the least cost from $x$ to $y$

        \begin{itemize}

          \item Then $x$ maintains distance vector $D_x=[D_x(y):\,y\in N]$

        \end{itemize}

      \item Iterative, asynchronous: each local iteration caused by:

        \begin{itemize}

          \item Local link cost change

          \item DV update message from neighbor

        \end{itemize}

      \item Distributed, self-stopping: each node notifies neighbors only when its DV changes

        \begin{itemize}

          \item Neighbors then notify their neighbors, only if necessary

          \item No notification received; no action taken

        \end{itemize}

    \end{itemize}

\end{itemize}

\end{document}

