%%%%%%%%%%%%%%%%%%%%%%%%%%%%%%%%%%%%%%%%%%%%%%%%%%%%%%%%%%%%%%%%%%%%%%%%%%%%%%%%%%%%%%%%%%%%%%%%%%%%%%%%%%%%%%%%%%%%%%%%%%%%%%%%%%%%%%%%%%%%%%%%%%%%%%%%%%%%%%%%%%%
% Written By Michael Brodskiy
% Class: Fundamentals of Networks
% Professor: E. Bernal Mor
%%%%%%%%%%%%%%%%%%%%%%%%%%%%%%%%%%%%%%%%%%%%%%%%%%%%%%%%%%%%%%%%%%%%%%%%%%%%%%%%%%%%%%%%%%%%%%%%%%%%%%%%%%%%%%%%%%%%%%%%%%%%%%%%%%%%%%%%%%%%%%%%%%%%%%%%%%%%%%%%%%%

\include{Includes.tex}

\title{The Network Layer: Control Plane}
\date{\today}
\author{Michael Brodskiy\\ \small Professor: E. Bernal Mor}

\begin{document}

\maketitle

\begin{itemize}

  \item Network-Layer Functions

    \begin{itemize}

      \item Forwarding (data plane)

      \item Routing: determine route taken by packets from source to destination (control plane)

        \begin{itemize}

          \item Two approaches to structuring a network control plane:

            \begin{itemize}

              \item Per-router plane (traditional)

              \item Software-defined

            \end{itemize}

        \end{itemize}

    \end{itemize}

  \item Per-Router Control Plane

    \begin{itemize}

      \item Individual routing algorithm components in each and every router interact in the control plane

    \end{itemize}

  \item Logically Centralized Control Plane (SDN)

    \begin{itemize}

      \item Remote controller computers, installs forwarding tables (aka flow tables) in routers)

    \end{itemize}

  \item Routing Protocols

    \begin{itemize}

      \item Routing protocol goal: determine ``good'' paths (equivalently, routes) from sending hosts to receiving hosts, through network of routers

        \begin{itemize}

          \item Path: sequence of routers that packets traverse from given initial source host to destination host

          \item ``Good'': least ``cost'', ``fastest'', ``least congested''

          \item Routing is a top networking challenge

        \end{itemize}

    \end{itemize}

  \item Graph Abstraction: Link Costs

    \begin{itemize}

      \item $c_{a,b}$ is the cost of a direct link connecting $a$ and $b$

        \begin{itemize}

          \item Cost is defined by network operator: could always be 1, or inversely related to link capacity, or proportional to length, etc.

        \end{itemize}

      \item The overall cost is a sum of all the costs from link to link

      \item The goal of a routing algorithm is to identify the least-cost path (aka shortest path) from sources to destination

      \item If all links have the same cost, the least-cost path is the path with the minimal number of links

    \end{itemize}

  \item Routing Algorithm Classification

    \begin{itemize}

      \item Centralized or global: all routers have complete topology, link cost info (``link state'' algorithms)

      \item Decentralized: iterative process of computation, exchange of info with neighbors (``distance vector'' algorithms)

      \item Static: routes change slowly over time

      \item Dynamic: routes change more quickly (periodic updates or in response to link cost changes)

    \end{itemize}

  \item Djikstra's Link-State Routing Algorithm

    \begin{itemize}

      \item Centralized: network topology and link costs known to all nodes

        \begin{itemize}

          \item Accomplished vie ``link state broadcast''

          \item All nodes have same info

        \end{itemize}

      \item Computes least cost paths from one node (``source'') to all other nodes

        \begin{itemize}

          \item Gives forwarding table for that node

        \end{itemize}

      \item Iterative: after $k$ iterations, know least cost path to $k$ destinations

    \end{itemize}

\end{itemize}

\end{document}

