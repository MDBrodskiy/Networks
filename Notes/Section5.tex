%%%%%%%%%%%%%%%%%%%%%%%%%%%%%%%%%%%%%%%%%%%%%%%%%%%%%%%%%%%%%%%%%%%%%%%%%%%%%%%%%%%%%%%%%%%%%%%%%%%%%%%%%%%%%%%%%%%%%%%%%%%%%%%%%%%%%%%%%%%%%%%%%%%%%%%%%%%%%%%%%%%
% Written By Michael Brodskiy
% Class: Fundamentals of Networks
% Professor: E. Bernal Mor
%%%%%%%%%%%%%%%%%%%%%%%%%%%%%%%%%%%%%%%%%%%%%%%%%%%%%%%%%%%%%%%%%%%%%%%%%%%%%%%%%%%%%%%%%%%%%%%%%%%%%%%%%%%%%%%%%%%%%%%%%%%%%%%%%%%%%%%%%%%%%%%%%%%%%%%%%%%%%%%%%%%

\documentclass[12pt]{article} 
\usepackage{alphalph}
\usepackage[utf8]{inputenc}
\usepackage[russian,english]{babel}
\usepackage{titling}
\usepackage{amsmath}
\usepackage{graphicx}
\usepackage{enumitem}
\usepackage{amssymb}
\usepackage[super]{nth}
\usepackage{everysel}
\usepackage{ragged2e}
\usepackage{geometry}
\usepackage{multicol}
\usepackage{fancyhdr}
\usepackage{cancel}
\usepackage{siunitx}
\usepackage{physics}
\usepackage{tikz}
\usepackage{mathdots}
\usepackage{yhmath}
\usepackage{cancel}
\usepackage{color}
\usepackage{array}
\usepackage{multirow}
\usepackage{gensymb}
\usepackage{tabularx}
\usepackage{extarrows}
\usepackage{booktabs}
\usepackage{lastpage}
\usepackage{float}
\usetikzlibrary{fadings}
\usetikzlibrary{patterns}
\usetikzlibrary{shadows.blur}
\usetikzlibrary{shapes}

\geometry{top=1.0in,bottom=1.0in,left=1.0in,right=1.0in}
\newcommand{\subtitle}[1]{%
  \posttitle{%
    \par\end{center}
    \begin{center}\large#1\end{center}
    \vskip0.5em}%

}
\usepackage{hyperref}
\hypersetup{
colorlinks=true,
linkcolor=blue,
filecolor=magenta,      
urlcolor=blue,
citecolor=blue,
}


\title{The Network Layer: Control Plane}
\date{\today}
\author{Michael Brodskiy\\ \small Professor: E. Bernal Mor}

\begin{document}

\maketitle

\begin{itemize}

  \item Network-Layer Functions

    \begin{itemize}

      \item Forwarding (data plane)

      \item Routing: determine route taken by packets from source to destination (control plane)

        \begin{itemize}

          \item Two approaches to structuring a network control plane:

            \begin{itemize}

              \item Per-router plane (traditional)

              \item Software-defined

            \end{itemize}

        \end{itemize}

    \end{itemize}

  \item Per-Router Control Plane

    \begin{itemize}

      \item Individual routing algorithm components in each and every router interact in the control plane

    \end{itemize}

  \item Logically Centralized Control Plane (SDN)

    \begin{itemize}

      \item Remote controller computers, installs forwarding tables (aka flow tables) in routers)

    \end{itemize}

  \item Routing Protocols

    \begin{itemize}

      \item Routing protocol goal: determine ``good'' paths (equivalently, routes) from sending hosts to receiving hosts, through network of routers

        \begin{itemize}

          \item Path: sequence of routers that packets traverse from given initial source host to destination host

          \item ``Good'': least ``cost'', ``fastest'', ``least congested''

          \item Routing is a top networking challenge

        \end{itemize}

    \end{itemize}

  \item Graph Abstraction: Link Costs

    \begin{itemize}

      \item $c_{a,b}$ is the cost of a direct link connecting $a$ and $b$

        \begin{itemize}

          \item Cost is defined by network operator: could always be 1, or inversely related to link capacity, or proportional to length, etc.

        \end{itemize}

      \item The overall cost is a sum of all the costs from link to link

      \item The goal of a routing algorithm is to identify the least-cost path (aka shortest path) from sources to destination

      \item If all links have the same cost, the least-cost path is the path with the minimal number of links

    \end{itemize}

  \item Routing Algorithm Classification

    \begin{itemize}

      \item Centralized or global: all routers have complete topology, link cost info (``link state'' algorithms)

      \item Decentralized: iterative process of computation, exchange of info with neighbors (``distance vector'' algorithms)

      \item Static: routes change slowly over time

      \item Dynamic: routes change more quickly (periodic updates or in response to link cost changes)

    \end{itemize}

  \item Djikstra's Link-State Routing Algorithm

    \begin{itemize}

      \item Centralized: network topology and link costs known to all nodes

        \begin{itemize}

          \item Accomplished vie ``link state broadcast''

          \item All nodes have same info

        \end{itemize}

      \item Computes least cost paths from one node (``source'') to all other nodes

        \begin{itemize}

          \item Gives forwarding table for that node

        \end{itemize}

      \item Iterative: after $k$ iterations, know least cost path to $k$ destinations

      \item Notation:

        \begin{itemize}

          \item $c_{x,y}$: direct link cost from node $x$ to $y$; $c_{x,y}=\infty$ if not direct neighbors
            
          \item $D(v)$: current estimate of cost of least-cost-path from source to destination $v$

          \item $p(v)$: predecessor node along path from source to $v$

          \item $N'$: set of nodes whose least-cost-path is definitively known

          \item Ties can exist, and are broken arbitrarily

          \item Construct least-cost-path tree by tracing predecessor nodes

        \end{itemize}

      \item Djikstra's Algorithm Complexity: $n$ nodes

        \begin{itemize}

          \item Each of $n$ iterations: need to check all the nodes, $w$, not in $N'$

          \item $n(n+1)/2$ comparisons: $O(n^2)$ complexity

          \item More efficient implementations possible $O(n\log(n))$

        \end{itemize}

      \item Oscillations possible: when link costs depend on traffic volume

    \end{itemize}

  \item Distance Vector Algorithm

    \begin{itemize}

      \item Based on Bellman-Ford (BF) Equation [dynamic programming]

      \item Let $d_x(y)$: cost of least-cost path from $x$ to $y$, then:

        $$d_x(y)=\text{min}_v(c_{x,v}+d_v(y))$$

      \item $D_x(y)$ is the estimate of the least cost from $x$ to $y$

        \begin{itemize}

          \item Then $x$ maintains distance vector $D_x=[D_x(y):\,y\in N]$

        \end{itemize}

      \item Iterative, asynchronous: each local iteration caused by:

        \begin{itemize}

          \item Local link cost change

          \item DV update message from neighbor

        \end{itemize}

      \item Distributed, self-stopping: each node notifies neighbors only when its DV changes

        \begin{itemize}

          \item Neighbors then notify their neighbors, only if necessary

          \item No notification received; no action taken

        \end{itemize}

    \end{itemize}

  \item Distance Vector: Link Cost Changes

    \begin{itemize}

      \item Node detects local link cost change

      \item Updates routing info, recalculates local DV

      \item If DV changes, notify neighbors

      \item ``Good news travels fast'' and ``Bad news travels slow'' (count to infinity problem)

    \end{itemize}

  \item Comparison of LS and DV Algorithms

    \begin{itemize}

      \item Message Complexity

        \begin{itemize}

          \item LS: $n$ nodes, $E$ links, $O(nE)$ messages sent

          \item DV: exchange between neighbors only during convergence time

        \end{itemize}

      \item Speed of Convergence

        \begin{itemize}

          \item LS: $O(n^2)$ algorithm requires $O(nE)$ messages

            \begin{itemize}

              \item May have oscillations

            \end{itemize}

          \item DV: convergence time varies

            \begin{itemize}

              \item May be routing loops

              \item Count-to-infinity problem

            \end{itemize}

        \end{itemize}

      \item Robustness: what happens if router malfunctions or is compromised

        \begin{itemize}

          \item LS:

            \begin{itemize}

              \item Router can advertise incorrect link cost

              \item Each router computes only its own table

            \end{itemize}

          \item DV:

            \begin{itemize}

              \item DV router can advertise incorrect path cost

                \begin{itemize}

                  \item ``I have a really low-cost path to everywhere'' is called black-holing

                \end{itemize}

              \item Each router's table used by others

              \item Errors propagate through network

            \end{itemize}

        \end{itemize}

    \end{itemize}

  \item Internet Approach to Scalable Routing

    \begin{itemize}

      \item Aggregate routers into regions known as ``Autonomous Systems'' (AS, aka ``domains'')

      \item Intra-AS routing (aka ``intra-domain''): routing among routers in the same AS (``network'')

        \begin{itemize}

          \item All routers in AS must run the same intra-domain protocol

          \item Routers in different AS can run different intra-domain routing

          \item Gateway router: at ``edge'' of its own AS, it has link(s) to router(s) in other ASs

        \end{itemize}

    \end{itemize}

  \item Intra-AS Routing: Routing Within an AS

    \begin{itemize}

      \item Most common intra-AS routing protocols

        \begin{itemize}

          \item RIP: Routing Information Protocol

            \begin{itemize}

              \item Classic DV: DVs exchanged every 30 seconds

              \item No longer widely used

            \end{itemize}

          \item EIGRP: Enhanced Interior Gateway Routing Protocol

            \begin{itemize}

              \item Formerly Cisco-proprietary for decades (became open in 2013)

              \item DV-based

            \end{itemize}

          \item OSPF: Open Shortest Path First

            \begin{itemize}

              \item Link-state routing

            \end{itemize}

          \item IS-IS Protocol (Intermediate System to Intermediate System) is an ISO standard (not RFC standard) that is essentially same as OSPF

        \end{itemize}

    \end{itemize}

  \item OSPF Routing

    \begin{itemize}

      \item ``Open'': publicly available

      \item Classic Link-State

        \begin{itemize}

          \item Each router floods OSPF link-state advertisements (directly over IP rather than using TCP/UDP) to all other routers in entire AS

          \item Multiple link costs metrics possible: link capacity, delay

          \item Each router has full topology and uses Djikstra's algorithm to compute forwarding table

        \end{itemize}

      \item Security: all OSPF messages authenticated (to prevent malicious intrusion)

    \end{itemize}

  \item Hierarchical OSPF

    \begin{itemize}

      \item Two-level hierarchy: local area, backbone

        \begin{itemize}

          \item Link-state advertisements flooded only in area, or backbone

          \item Each node has detailed area topology; only knows direction to reach other destinations

        \end{itemize}

    \end{itemize}

  \item Internet Inter-AS Routing: BGP

    \begin{itemize}

      \item BGP (Border Gateway Protocol): the de facto inter-domain routing protocol

        \begin{itemize}

          \item ``Glue that holds the Internet control plane together''

          \item Uses an algorithm in the vein of distance vector routing

        \end{itemize}

      \item Allows network to advertise its existence, and the destinations it can reach, to rest of internet: ``I am here, here is who I can reach, and how''

        \begin{itemize}

          \item Advertise prefixes (subnet or collection of subnets)

        \end{itemize}

      \item BGP provides each AS a means to:

        \begin{itemize}

          \item eBGP (external BGP): obtain subnet reachability information from neighboring ASes

          \item iBGP (internal BGP): propagate reachability information to all AS-internal routers

          \item Determine ``good'' routes to other networks based on reachability information and policy

        \end{itemize}

    \end{itemize}

  \item BGP Basics

    \begin{itemize}

      \item BGP session: two BGP routers (``peers'') exchange BGP messages over semi-permanent TCP connection

      \item BGP is a ``path vector'' protocol: advertises paths to different destination network prefixes

    \end{itemize}

  \item Path Attributs and BGP Routes

    \begin{itemize}

      \item BGP advertised route: prefix + attributes

        \begin{itemize}

          \item Prefix: destination advertised

          \item Two important attributes:

            \begin{enumerate}

              \item AS-PATH: list of ASs through which prefix advertisement has passed

              \item NEXT-HOP: indicates specific internal-AS router to next-hop AS

            \end{enumerate}

        \end{itemize}

      \item Policy-based Routing

        \begin{itemize}

          \item Gateway receiving route advertisement uses import policy to accept/decline a path

        \end{itemize}

    \end{itemize}

\end{itemize}

\end{document}

