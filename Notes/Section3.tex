%%%%%%%%%%%%%%%%%%%%%%%%%%%%%%%%%%%%%%%%%%%%%%%%%%%%%%%%%%%%%%%%%%%%%%%%%%%%%%%%%%%%%%%%%%%%%%%%%%%%%%%%%%%%%%%%%%%%%%%%%%%%%%%%%%%%%%%%%%%%%%%%%%%%%%%%%%%%%%%%%%%
% Written By Michael Brodskiy
% Class: Fundamentals of Networks
% Professor: E. Bernal Mor
%%%%%%%%%%%%%%%%%%%%%%%%%%%%%%%%%%%%%%%%%%%%%%%%%%%%%%%%%%%%%%%%%%%%%%%%%%%%%%%%%%%%%%%%%%%%%%%%%%%%%%%%%%%%%%%%%%%%%%%%%%%%%%%%%%%%%%%%%%%%%%%%%%%%%%%%%%%%%%%%%%%

\documentclass[12pt]{article} 
\usepackage{alphalph}
\usepackage[utf8]{inputenc}
\usepackage[russian,english]{babel}
\usepackage{titling}
\usepackage{amsmath}
\usepackage{graphicx}
\usepackage{enumitem}
\usepackage{amssymb}
\usepackage[super]{nth}
\usepackage{everysel}
\usepackage{ragged2e}
\usepackage{geometry}
\usepackage{multicol}
\usepackage{fancyhdr}
\usepackage{cancel}
\usepackage{siunitx}
\usepackage{physics}
\usepackage{tikz}
\usepackage{mathdots}
\usepackage{yhmath}
\usepackage{cancel}
\usepackage{color}
\usepackage{array}
\usepackage{multirow}
\usepackage{gensymb}
\usepackage{tabularx}
\usepackage{extarrows}
\usepackage{booktabs}
\usepackage{lastpage}
\usepackage{float}
\usetikzlibrary{fadings}
\usetikzlibrary{patterns}
\usetikzlibrary{shadows.blur}
\usetikzlibrary{shapes}

\geometry{top=1.0in,bottom=1.0in,left=1.0in,right=1.0in}
\newcommand{\subtitle}[1]{%
  \posttitle{%
    \par\end{center}
    \begin{center}\large#1\end{center}
    \vskip0.5em}%

}
\usepackage{hyperref}
\hypersetup{
colorlinks=true,
linkcolor=blue,
filecolor=magenta,      
urlcolor=blue,
citecolor=blue,
}


\title{The Transport Layer}
\date{\today}
\author{Michael Brodskiy\\ \small Professor: E. Bernal Mor}

\begin{document}

\maketitle

\begin{itemize}

  \item Transport Services and Protocols

    \begin{itemize}

      \item Provide logical communication between application processes running on different hosts

      \item Transport protocols actions in end systems:

        \begin{itemize}

          \item Sender: breaks application messages into segments, passes to Network layer

          \item Receiver: reassembles segments into messages, passes to Application layer

        \end{itemize}

      \item Two transport protocols available to internet applications

        \begin{enumerate}

          \item TCP

          \item UDP

        \end{enumerate}

    \end{itemize}

  \item Transport vs. Network Layer

    \begin{itemize}

      \item Network layer: logical communication between two hosts

      \item Transport layer: logical communication between processes

        \begin{itemize}

          \item Relies on, enhances, network layer services

        \end{itemize}

    \end{itemize}

  \item Two Internet Transport Protocols

    \begin{itemize}

      \item TCP: Transmission Control Protocol

        \begin{itemize}

          \item Reliable, in-order delivery

          \item Congestion Control

          \item Flow Control

          \item Connection set-up

        \end{itemize}

      \item UDP: User Datagram Protocol

        \begin{itemize}
            
          \item Unreliable, unordered delivery

          \item No-frills extension of ``best-effort'' IP

        \end{itemize}

      \item Services not available:

        \begin{itemize}

          \item Delay guarantees

          \item Throughput guarantees

        \end{itemize}

    \end{itemize}

  \item Multiplexing/Demultiplexing

    \begin{itemize}

      \item Multiplexing at sender: Handle data from multiple sockets, add transport header (later used for demultiplexing)

      \item How demultiplexing works

        \begin{itemize}

          \item Host receives IP packets

            \begin{itemize}

              \item Each packet has source IP address, destination IP address

              \item Each packet carries one transport-layer segment

              \item Each segment has source, destination port number

            \end{itemize}

          \item Host uses IP address and port numbers to direct segment to appropriate socket

        \end{itemize}

      \item Connectionless Demultiplexing

        \begin{itemize}

          \item Create a socket in the client, the Transport layer automatically assigns a host-local port number to the socket

          \item When data is sent into UDP socket, must specify

            \begin{itemize}
                
              \item Destination IP address
                
              \item Destination port number

            \end{itemize}

          \item When a host receives UDP segment, the Transport layer:

            \begin{itemize}

              \item Checks destination port number in segment

              \item Directs UDP segment to socket with that port number

            \end{itemize}

          \item IP datagrams with same destination port number but different source IP addresses and/or source port numbers will be directed to same socket at destination

        \end{itemize}

      \item Connection-Oriented Demultiplexing

        \begin{itemize}

          \item TCP socket identified by 4-tuple:

            \begin{itemize}

              \item Source IP address

              \item Source port number

              \item Destination IP address

              \item Destination port number

            \end{itemize}

          \item Demultiplexing receiver users all four values to direct segment to appropriate socket

          \item A server may support simultaneous TCP sockets:

            \begin{itemize}

              \item Each socket identified by its own 4-tuple

              \item Each socket associated with a different connecting client

            \end{itemize}

          \item Note: the TCP server has a welcoming socket

            \begin{itemize}

              \item Each time a client initiates a TCP connection to the server, a new socket is created for this connection

              \item To support $n$ simultaneous connections, the server would need $n+1$ sockets

            \end{itemize}

        \end{itemize}

    \end{itemize}

  \item Connectionless Transport: UDP

    \begin{itemize}

      \item ``No frills'', ``bare bones'' Internet transport protocol

      \item ``Best effort'' service, UDP segments may be:

        \begin{itemize}

          \item Lost

          \item Delivered out-of-order to application

        \end{itemize}

      \item Connectionless:

        \begin{itemize}

          \item No handshaking between UDP sender, receiver

          \item Each UDP segment handled independently of others

        \end{itemize}

      \item Why is there a UDP?

        \begin{itemize}

          \item No connection establishment (which can add RTT delay)

          \item Simple: no connection state at sender, receiver

          \item Small header size

          \item No congestion control

            \begin{itemize}

              \item UDP can blast away as fast as desired

              \item It can function in the face of congestion

            \end{itemize}

        \end{itemize}

      \item UDP used in:

        \begin{itemize}

          \item Streaming multimedia apps (loss tolerant, rate sensitive)

          \item DNS

          \item HTTP/3

        \end{itemize}

      \item If reliable transfer or other services needed over UDP (like in HTTP/3)

        \begin{itemize}

          \item Add needed reliability at application layer

          \item Add congestion control at application layer

        \end{itemize}

    \end{itemize}

  \item Internet Checksum

    \begin{itemize}

      \item Goal: detect ``errors'' (\textit{e}.\textit{g}.\ flipped bits) in received segment —  optional if UDP segment is encapsulated in IPv4 packet: carries all-zeros if unused

      \item Sender:

        \begin{itemize}

          \item Treat segment contents, including UDP header fields and IP addresses, as sequence of 16-bit words

          \item Checksum: ones complement of the ones complement sum of all 16-bit words

          \item Sender puts checksum value into UDP checksum field

        \end{itemize}

      \item Receiver — error detected?

        \begin{itemize}

          \item All 16-bit words are added, including checksum

          \item Result = $1111111111111111\to$ no error 

            \begin{itemize}

              \item But maybe errors, nonetheless?

            \end{itemize}

          \item Result $\neq1111111111111111\to$ Error

            \begin{itemize}

              \item Do not recover from the error: discard segment or pass damaged data with a warning

            \end{itemize}

        \end{itemize}

    \end{itemize}

  \item Principles of Reliable Data Transfer

    \begin{itemize}

      \item Consider only unidirectional data transfer

      \item But control information will flow in both directions

      \item Incrementally develop sender, receiver sides of reliable data transfer protocol (rdt)

    \end{itemize}

  \item Components of Error Recovery

    \begin{itemize}

      \item Checksum to detect bit errors

      \item Acknowledgments (ACKs): receiver explicitly tells sender the packet received is OK

      \item Negative Acknowledgments (NAKs): receiver explicitly tells sender that pack had errors

      \item Sender retransmits packet on receipt of NAK

      \item What happens if ACK/NAK is corrupted?

        \begin{itemize}

          \item Sender doesn't know what happened at receiver

          \item Can't just retransmit, possible duplicate

        \end{itemize}

      \item Sender:

        \begin{itemize}

          \item Sequence number added to packet

          \item Two sequence numbers will suffice

          \item Must check if received ACK/NAK corrupted

          \item Sender must ``remember'' whether last sent packet had sequence number of 0 or 1

        \end{itemize}

      \item Receiver:

        \begin{itemize}

          \item Must check if received packet is duplicate

            \begin{itemize}

              \item Receiver must remember whether 0 or 1 is expected packet sequence number

            \end{itemize}

          \item If duplicate, send ACK to force sender to move on

          \item Note: receiver can not know if its last NAK/ACK received ok at sender

        \end{itemize}

      \item A similar protocol without NAKs may be developed:

        \begin{itemize}

          \item Instead of NAK, receiver sends ACK for last packet received OK

            \begin{itemize}

              \item Receiver must explicitly include in the ACK the sequence number of packet being ACKed

            \end{itemize}

          \item Duplicate ACK at sender results in same action as NAK: retransmit current packet

          \item TCP uses this approach to be NAK-free

        \end{itemize}

      \item This protocol may be further developed:

        \begin{itemize}

          \item Approach: sender waits ``reasonable'' amount of time for ACK

          \item Only retransmit if no expected ACK received in this time

          \item If packet (or ACK) just delayed (not lost):

            \begin{itemize}

              \item Retransmission will be duplicate, but sequence number already handles this

              \item Receiver must specify sequence number of packet being ACKed

            \end{itemize}

          \item Requires countdown timer to interrupt after “reasonable” amount of time (timeout)

        \end{itemize}

    \end{itemize}

  \item Utility

    \begin{itemize}

      \item For a ``Stop and Wait'' approach, the utility can be defined as:

        $$U_{sender}=\frac{L/R}{RTT+(L/R)}$$

      \item Thus, the performance of stop and wait would not be food if the RTT is high

      \item Stop and wait limits performance of underlying infrastructure

    \end{itemize}

  \item Pipelined Protocols

    \begin{itemize}

      \item Pipelining: sender allows multiple, ``in-flight'', yet-to-be-acknowledged packets

        \begin{itemize}

          \item Range of sequence numbers must be increaseL $k$-bit sequence number in packet header

          \item Buffering at sender and/or receiver

        \end{itemize}

      \item Two generic forms of pipelined protocols: Go-Back-N (GBN) and selective repeat

    \end{itemize}

  \item Go-Back-N

    \begin{itemize}

      \item A ``sliding window'' protocol

      \item Sender window: $N$ consecutive sequence numbers allowed for sent, unacknowledged packets

        \begin{itemize}

          \item Sender window size is $N$

        \end{itemize}

      \item Cumulative ACK: ACK($n$) $\to$ Acknowledges all packets up to $N$, including sequence number $n$

        \begin{itemize}

          \item On receiving ACK($n$) $\to$ move window forward to begin at $n+1$

        \end{itemize}

    \end{itemize}

  \item Selective Repeat

    \begin{itemize}

      \item Sender Window

        \begin{itemize}

          \item $N$ consecutive sequence numbers

          \item Limits sequence numbers of sent, unacknowledged packets

          \item Sender window size is $N$

        \end{itemize}

      \item Sender maintains a timer for each unACKed packet

      \item Sender times-out and retrsnmits individually unACKed packets

      \item Receiver Window

        \begin{itemize}

          \item $N$ consecutive sequence number

          \item Limits sequence numbers of received packets that are accepted

          \item Receiver window size is $N$

        \end{itemize}

      \item Buffers packets, as needed for eventual in-order delivery to upper layer

      \item Receiver individually acknowledges all correctly received packets

    \end{itemize}

  \item Connection-Oriented Transport: TCP

    \begin{itemize}

      \item Point-to-point: one sender, one receiver

      \item Reliable, in-order byte stream:

        \begin{itemize}

          \item No ``message boundaries''

        \end{itemize}

      \item Full duplex data:

        \begin{itemize}

          \item Bi-directional data flow in same connection

          \item MSS: maximum segment size

        \end{itemize}

      \item Cumulative ACKs

      \item Pipelining:

        \begin{itemize}

          \item TCP congestion and flow control set the window size

        \end{itemize}

      \item Connection-oriented

        \begin{itemize}

          \item Handshaking (exchange of control messages) initializes sender and receiver state before data exchange

        \end{itemize}

      \item Flow control

        \begin{itemize}

          \item Sender will not overwhelm receiver

        \end{itemize}

    \end{itemize}

\end{itemize}

\end{document}

