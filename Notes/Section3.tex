%%%%%%%%%%%%%%%%%%%%%%%%%%%%%%%%%%%%%%%%%%%%%%%%%%%%%%%%%%%%%%%%%%%%%%%%%%%%%%%%%%%%%%%%%%%%%%%%%%%%%%%%%%%%%%%%%%%%%%%%%%%%%%%%%%%%%%%%%%%%%%%%%%%%%%%%%%%%%%%%%%%
% Written By Michael Brodskiy
% Class: Fundamentals of Networks
% Professor: E. Bernal Mor
%%%%%%%%%%%%%%%%%%%%%%%%%%%%%%%%%%%%%%%%%%%%%%%%%%%%%%%%%%%%%%%%%%%%%%%%%%%%%%%%%%%%%%%%%%%%%%%%%%%%%%%%%%%%%%%%%%%%%%%%%%%%%%%%%%%%%%%%%%%%%%%%%%%%%%%%%%%%%%%%%%%

\documentclass[12pt]{article} 
\usepackage{alphalph}
\usepackage[utf8]{inputenc}
\usepackage[russian,english]{babel}
\usepackage{titling}
\usepackage{amsmath}
\usepackage{graphicx}
\usepackage{enumitem}
\usepackage{amssymb}
\usepackage[super]{nth}
\usepackage{everysel}
\usepackage{ragged2e}
\usepackage{geometry}
\usepackage{multicol}
\usepackage{fancyhdr}
\usepackage{cancel}
\usepackage{siunitx}
\usepackage{physics}
\usepackage{tikz}
\usepackage{mathdots}
\usepackage{yhmath}
\usepackage{cancel}
\usepackage{color}
\usepackage{array}
\usepackage{multirow}
\usepackage{gensymb}
\usepackage{tabularx}
\usepackage{extarrows}
\usepackage{booktabs}
\usepackage{lastpage}
\usepackage{float}
\usetikzlibrary{fadings}
\usetikzlibrary{patterns}
\usetikzlibrary{shadows.blur}
\usetikzlibrary{shapes}

\geometry{top=1.0in,bottom=1.0in,left=1.0in,right=1.0in}
\newcommand{\subtitle}[1]{%
  \posttitle{%
    \par\end{center}
    \begin{center}\large#1\end{center}
    \vskip0.5em}%

}
\usepackage{hyperref}
\hypersetup{
colorlinks=true,
linkcolor=blue,
filecolor=magenta,      
urlcolor=blue,
citecolor=blue,
}


\title{The Transport Layer}
\date{\today}
\author{Michael Brodskiy\\ \small Professor: E. Bernal Mor}

\begin{document}

\maketitle

\begin{itemize}

  \item Transport Services and Protocols

    \begin{itemize}

      \item Provide logical communication between application processes running on different hosts

      \item Transport protocols actions in end systems:

        \begin{itemize}

          \item Sender: breaks application messages into segments, passes to Network layer

          \item Receiver: reassembles segments into messages, passes to Application layer

        \end{itemize}

      \item Two transport protocols available to internet applications

        \begin{enumerate}

          \item TCP

          \item UDP

        \end{enumerate}

    \end{itemize}

  \item Transport vs. Network Layer

    \begin{itemize}

      \item Network layer: logical communication between two hosts

      \item Transport layer: logical communication between processes

        \begin{itemize}

          \item Relies on, enhances, network layer services

        \end{itemize}

    \end{itemize}

  \item Two Internet Transport Protocols

    \begin{itemize}

      \item TCP: Transmission Control Protocol

        \begin{itemize}

          \item Reliable, in-order delivery

          \item Congestion Control

          \item Flow Control

          \item Connection set-up

        \end{itemize}

      \item UDP: User Datagram Protocol

        \begin{itemize}
            
          \item Unreliable, unordered delivery

          \item No-frills extension of ``best-effort'' IP

        \end{itemize}

      \item Services not available:

        \begin{itemize}

          \item Delay guarantees

          \item Throughput guarantees

        \end{itemize}

    \end{itemize}

  \item Multiplexing/Demultiplexing

    \begin{itemize}

      \item Multiplexing at sender: Handle data from multiple sockets, add transport header (later used for demultiplexing)

      \item How demultiplexing works

        \begin{itemize}

          \item Host receives IP packets

            \begin{itemize}

              \item Each packet has source IP address, destination IP address

              \item Each packet carries one transport-layer segment

              \item Each segment has source, destination port number

            \end{itemize}

          \item Host uses IP address and port numbers to direct segment to appropriate socket

        \end{itemize}

      \item Connectionless Demultiplexing

        \begin{itemize}

          \item Create a socket in the client, the Transport layer automatically assigns a host-local port number to the socket

          \item When data is sent into UDP socket, must specify

            \begin{itemize}
                
              \item Destination IP address
                
              \item Destination port number

            \end{itemize}

          \item When a host receives UDP segment, the Transport layer:

            \begin{itemize}

              \item Checks destination port number in segment

              \item Directs UDP segment to socket with that port number

            \end{itemize}

          \item IP datagrams with same destination port number but different source IP addresses and/or source port numbers will be directed to same socket at destination

        \end{itemize}

      \item Connection-Oriented Demultiplexing

        \begin{itemize}

          \item TCP socket identified by 4-tuple:

            \begin{itemize}

              \item Source IP address

              \item Source port number

              \item Destination IP address

              \item Destination port number

            \end{itemize}

          \item Demultiplexing receiver users all four values to direct segment to appropriate socket

          \item A server may support simultaneous TCP sockets:

            \begin{itemize}

              \item Each socket identified by its own 4-tuple

              \item Each socket associated with a different connecting client

            \end{itemize}

          \item Note: the TCP server has a welcoming socket

            \begin{itemize}

              \item Each time a client initiates a TCP connection to the server, a new socket is created for this connection

              \item To support $n$ simultaneous connections, the server would need $n+1$ sockets

            \end{itemize}

        \end{itemize}

    \end{itemize}

  \item Connectionless Transport: UDP

    \begin{itemize}

      \item ``No frills'', ``bare bones'' Internet transport protocol

      \item ``Best effort'' service, UDP segments may be:

        \begin{itemize}

          \item Lost

          \item Delivered out-of-order to application

        \end{itemize}

      \item Connectionless:

        \begin{itemize}

          \item No handshaking between UDP sender, receiver

          \item Each UDP segment handled independently of others

        \end{itemize}

      \item Why is there a UDP?

        \begin{itemize}

          \item No connection establishment (which can add RTT delay)

          \item Simple: no connection state at sender, receiver

          \item Small header size

          \item No congestion control

            \begin{itemize}

              \item UDP can blast away as fast as desired

              \item It can function in the face of congestion

            \end{itemize}

        \end{itemize}

      \item UDP used in:

        \begin{itemize}

          \item Streaming multimedia apps (loss tolerant, rate sensitive)

          \item DNS

          \item HTTP/3

        \end{itemize}

      \item If reliable transfer or other services needed over UDP (like in HTTP/3)

        \begin{itemize}

          \item Add needed reliability at application layer

          \item Add congestion control at application layer

        \end{itemize}

    \end{itemize}

\end{itemize}

\end{document}

