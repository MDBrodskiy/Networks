%%%%%%%%%%%%%%%%%%%%%%%%%%%%%%%%%%%%%%%%%%%%%%%%%%%%%%%%%%%%%%%%%%%%%%%%%%%%%%%%%%%%%%%%%%%%%%%%%%%%%%%%%%%%%%%%%%%%%%%%%%%%%%%%%%%%%%%%%%%%%%%%%%%%%%%%%%%%%%%%%%%
% Written By Michael Brodskiy
% Class: Fundamentals of Networks
% Professor: E. Bernal Mor
%%%%%%%%%%%%%%%%%%%%%%%%%%%%%%%%%%%%%%%%%%%%%%%%%%%%%%%%%%%%%%%%%%%%%%%%%%%%%%%%%%%%%%%%%%%%%%%%%%%%%%%%%%%%%%%%%%%%%%%%%%%%%%%%%%%%%%%%%%%%%%%%%%%%%%%%%%%%%%%%%%%

\documentclass[12pt]{article} 
\usepackage{alphalph}
\usepackage[utf8]{inputenc}
\usepackage[russian,english]{babel}
\usepackage{titling}
\usepackage{amsmath}
\usepackage{graphicx}
\usepackage{enumitem}
\usepackage{amssymb}
\usepackage[super]{nth}
\usepackage{everysel}
\usepackage{ragged2e}
\usepackage{geometry}
\usepackage{multicol}
\usepackage{fancyhdr}
\usepackage{cancel}
\usepackage{siunitx}
\usepackage{physics}
\usepackage{tikz}
\usepackage{mathdots}
\usepackage{yhmath}
\usepackage{cancel}
\usepackage{color}
\usepackage{array}
\usepackage{multirow}
\usepackage{gensymb}
\usepackage{tabularx}
\usepackage{extarrows}
\usepackage{booktabs}
\usepackage{lastpage}
\usepackage{float}
\usetikzlibrary{fadings}
\usetikzlibrary{patterns}
\usetikzlibrary{shadows.blur}
\usetikzlibrary{shapes}

\geometry{top=1.0in,bottom=1.0in,left=1.0in,right=1.0in}
\newcommand{\subtitle}[1]{%
  \posttitle{%
    \par\end{center}
    \begin{center}\large#1\end{center}
    \vskip0.5em}%

}
\usepackage{hyperref}
\hypersetup{
colorlinks=true,
linkcolor=blue,
filecolor=magenta,      
urlcolor=blue,
citecolor=blue,
}


\title{The Link Layer}
\date{\today}
\author{Michael Brodskiy\\ \small Professor: E. Bernal Mor}

\begin{document}

\maketitle

\begin{itemize}

  \item Link Layer: Introduction

    \begin{itemize}

      \item Terminology

        \begin{itemize}

          \item Hosts, routers $\to$ nodes

          \item Communication channels that connect adjacent nodes along communication path $\to$ links

            \begin{itemize}

              \item Wired links

              \item Wireless links

            \end{itemize}

          \item Over a given link, the transmitting node encapsulates the network-layer packet in a link-layer frame

        \end{itemize}

      \item Link layer has responsibility of transferring network-layer packets from one node to a physically adjacent node over a link

    \end{itemize}

  \item Link Layer: Context

    \begin{itemize}

      \item Packets transferred by different link protocols over different links

        \begin{itemize}

          \item WiFi on first link

          \item Ethernet on next link

          \item Etcetera

        \end{itemize}

      \item Each link protocol provides different services

    \end{itemize}

  \item Link Layer Services

    \begin{itemize}

      \item Framing

        \begin{itemize}

          \item Encapsulate packet into frame, adding header and maybe trailer

          \item Addressing: ``MAC'' addresses used in frame headers to identify transmitter/receiver node $\to$ different from IP Address

        \end{itemize}

      \item Link access

        \begin{itemize}

          \item Medium access control (MAC) protocol specifies the rules by which a frame is transmitted onto the link

        \end{itemize}

      \item Flow control

        \begin{itemize}

          \item Pacing between adjacent sending and receiving nodes

        \end{itemize}

      \item Reliable delivery between adjacent nodes

        \begin{itemize}

          \item We learned how to do this already (Transport layer)!

          \item Seldom used on low error rate links, for example: fiber, some twisted pairs

          \item Commonly used on high error rate links, like wireless ones

        \end{itemize}

      \item Error control

        \begin{itemize}

          \item Errors caused by signal attenuation, noise

          \item Error detection: receiver detects presence of errors

            \begin{itemize}

              \item Ask sender for retransmission or drops frame

            \end{itemize}

          \item Error correction: receiver identifies and corrects bit error(s) without resorting to retransmission

        \end{itemize}

    \end{itemize}

  \item Half-Duplex and Full-Duplex Links

    \begin{itemize}

      \item Unidirectional links (Simplex Links)

        \begin{itemize}

          \item Communication occurs in one direction only

        \end{itemize}

      \item Bidirectional links 

        \begin{itemize}

          \item Half-Duplex Link — Communication occurs in both directions, but not at same time

          \item Full-Duplex Link — Communication occurs in both directions at same time

        \end{itemize}

    \end{itemize}

  \item Where is the Link Layer Implemented?

    \begin{itemize}

      \item For the most part, link layer is implemented on a chip called the network adapter, aka a Network Interface Card (NIC)

        \begin{itemize}

          \item The NIC implements Link and Physical layers

          \item \textit{E}.\textit{g}. Ethernet card,  WiFi card or chip

        \end{itemize}

      \item NIC attaches into node’s system buses

      \item Link layer is implemented as a combination of hardware and software

        \begin{itemize}

          \item Hardware: NIC implements most of the functions

          \item Software: activating hardware controller, responds to controller interrupts, etc.

        \end{itemize}

    \end{itemize}

  \item Error Control

    \begin{itemize}

      \item EDC $\to$ Error Detection/Correction bits (redundant bits)

      \item D $\to$ Data protected by error control, may include header fields

      \item Error control is not 100\% reliable

        \begin{itemize}

          \item Error control technique may miss some errors; we want to keep the probability of missing the errors small

          \item Larger EDC field yields better detection and correction

          \item Error correction needs more redundant bits than error detection for same number of errors

        \end{itemize}

      \item Parity Checking

        \begin{itemize}

          \item Single bit parity: detect single bit errors

            \begin{itemize}

              \item Even parity: set parity bit so there is an even number of 1's

              \item Odd parity: set parity bit so there is an odd number of 1's

            \end{itemize}

          \item Two-dimensional bit parity: detect and correct single bit errors

            \begin{itemize}

              \item Even parity: no errors

            \end{itemize}

        \end{itemize}

      \item Cyclic Redundancy Check (CRC)

        \begin{itemize}

          \item $D$: $d$ data bits (given)

          \item $G$: generator, bit pattern of $r+1$ bits where MSB must be 1 $\to$ transmitter and receiver agree on $G$ (given)

          \item $R$: $r$ CRC bits, redundant bits

          \item Transmitter: choose $R$, such that $\langle D,R\rangle$ is exactly divisible by $G$ (modulo-2 arithmetic) $\to D\cdot2^r\text{XOR}R=nG$

          \item Receiver: knows $G$ and divides $\langle D,R\rangle$ by $G\to$ non-zero remainder: error detected!

            \begin{itemize}

              \item Can detect all burst errors less than $r+1$ bits

            \end{itemize}

          \item More powerful error-detection technique: widely used in practice (Ethernet,  WiFi)

        \end{itemize}

    \end{itemize}

  \item Types of Links

    \begin{itemize}

      \item Point-to=point link: a single sender at one end on a link and a single receive at the other end of the link

        \begin{itemize}

          \item PPP (Point-to-Point Protocol), switched Ethernet, etc.

        \end{itemize}

      \item Broadcast (shared medium) link: Multiple transmitting and receiving nodes all connected to the same, single shared broadcast link

        \begin{itemize}

          \item Need to handle multiple access problem (classic Ethernet, 4G/5G, WiFi, etc.)

        \end{itemize}

    \end{itemize}

  \item Multiple Access Problem

    \begin{itemize}

      \item Multiple access problem: how to coordinate the access of multiple transmitting and receiving nodes to a single, shared broadcast channel

      \item Two or more simultaneous transmissions by nodes $\to$ interference

        \begin{itemize}

          \item Collision: if a node receives two or more signals at the same time (collision happens in the receiver)

        \end{itemize}

      \item MAC (Medium Access Control) Protocol

        \begin{itemize}

          \item Distributed algorithm that coordinates the frame transmissions of many nodes into the broadcast channel

          \item Determines how nodes share channel and when nodes can transmit

        \end{itemize}

        \begin{itemize}

          \item Communication about channel sharing must use channel itself! 

          \item No out-of-band channel for coordination

        \end{itemize}

    \end{itemize}

  \item An Ideal MAC Protocol

    \begin{itemize}

      \item Given: broadcast channel of rate $R$ bps

      \item Desirable Characteristics:

        \begin{enumerate}

          \item When one node wants to transmit, it can send at rate $R$

          \item When $M$ nodes want to transmit, each can send an average rate $R/M$

          \item Fully Decentralized

            \begin{itemize}

              \item No special node to coordinate transmissions

              \item No synchronizations of clocks, slots

            \end{itemize}

          \item Simple

        \end{enumerate}

    \end{itemize}

  \item MAC Protocols Taxonomy

    \begin{itemize}

      \item Three Broad Classes:

        \begin{enumerate}

          \item Channel Partitioning

            \begin{itemize}

              \item Divide channel into smaller ``pieces'' (time slots, frequency, code)

              \item Collision free: allocate piece of node for exclusive use to avoid collisions

            \end{itemize}

          \item Random Access

            \begin{itemize}

              \item Channel not divided, allow collisions

              \item ``Recover'' from collisions

            \end{itemize}

          \item Turn-Taking

            \begin{itemize}

              \item Nodes take turns: tightly coordinate shared access to avoid collisions

            \end{itemize}

        \end{enumerate}

    \end{itemize}

  \item Channel Partitioning: FDMA

    \begin{itemize}

      \item FDMA: Frequency Division Multiple Access

      \item Frequency spectrum of the channel is divided into $N$ frequency bands (each with transmission rate $R/N$ bps)

      \item Each node is assigned a fixed frequency band

      \item Advantages: Avoids collisions, and divides the capacity link fairly

      \item Drawback: Unused link capacity if frequency band goes idle

    \end{itemize}

  \item Channel Partitioning: TDMA

    \begin{itemize}

      \item TDMA: Time Division Multiple Access

      \item Access to channel in ``rounds'' $\to$ time divided in $N$ slots

      \item Each node gets fixed time slot in each round (node average transmission rate $R/N$ bps)

      \item Advantages: it avoids collisions, and it divides the link capacity fairly

      \item Drawback: unused link capacity if slots fo idle

    \end{itemize}

  \item Random Access Protocols

    \begin{itemize}

      \item A transmitting node always transmits at full channel rate, $R$ bps

      \item Two or more transmitting nodes create a collision

      \item Random access MAC protcool specifics:

        \begin{itemize}

          \item How to detect collisions

          \item How to recover from collisions (like via delayed transmissions)

        \end{itemize}

    \end{itemize}

  \item Slotted ALOHA

    \begin{itemize}

      \item Assumptions:

        \begin{itemize}

          \item All frames same size

          \item Time divided into equal size slots

            \begin{itemize}

              \item Slot duration: time to transmit one frame, $t_l$

            \end{itemize}

          \item Nodes start to transmit only at the beginning of slot

          \item Nodes are synchronized

          \item If 2 or more nodes transmit in slot, all nodes detect collision

        \end{itemize}

      \item Operation:

        \begin{itemize}

          \item When node obtains fresh frame, transmits in next slot

            \begin{itemize}

              \item If no collision: node can send new frame in next slot

              \item If collision: node retransmits frame in each subsequent slot with probability $p$ until success

            \end{itemize}

        \end{itemize}

    \end{itemize}

  \item Slotted ALOHA: Efficiency

    \begin{itemize}

      \item Efficiency: long-run fraction of successful slots (many nodes, all with many frames to send)

      \item Suppose $N$ nodes with many frames to send, each transmits in slot with probability $p$

        \begin{itemize}

          \item Probability that a given node has success in a slot: $p(1-p)^{N-1}$

          \item Probability that any node has a success: $Np(1-p)^{N-1}$

          \item Max efficiency: find $p^*$ that maximizes $Np(1-p)^{N-1}$

          \item For many nodes, take limit of $Np^*(1-p^*)^{N-1}$ as $N$ goes to infinity gives:

            $$\text{Max Efficiency: }\frac{1}{e}=.37$$

        \end{itemize}

    \end{itemize}

  \item Pure Aloha

    \begin{itemize}

      \item Unslotted Aloha: simpler, no synchronization (no time slots)

      \item Consider that $t_f$ is the frame transmission time

      \item When node obtains fresh frame, transmit immediately
        
      \item Collision: retransmit with probability $p$ immediately, repeat every $t_f$ until the frame is transmitted

      \item Collision probability increases:

        \begin{itemize}

          \item Frame sent at $t_0$ collides with other frames sent in $[t_0-t_f, t_0+t_f]$

        \end{itemize}

    \end{itemize}

  \item Pure Aloha Efficiency

    \begin{itemize}

      \item Efficiency at many nodes is $.18$, even worse than slotted aloha

    \end{itemize}

  \item Carrier Sense Multiple Access (CSMA)

    \begin{itemize}

      \item CSMA: listen before transmitting

        \begin{itemize}

          \item If channel sensed idle: transmit entire frame

          \item If channel sensed busy: defer transmission

        \end{itemize}

      \item CSMA/CD: CSMA with Collision Detection

        \begin{itemize}

          \item Collisions detected within short time

          \item Colliding transmissions aborted, reducing channel wastage

          \item After collision, wait a random time before repeating the CSMA/CD cycle

        \end{itemize}

    \end{itemize}

  \item CSMA: Collisions

    \begin{itemize}

      \item Collisions can still occur with carrier sensing:

        \begin{itemize}

          \item Propagation delay: two nodes may not heard each other's just-started transmission

          \item The longer the propagation delay from one node to another, the larger the probability that a node is not able to sense a transmission that has already begun at another node

          \item Distance and propagation delay play a crucial role in determining collisoon probability

        \end{itemize}

      \item Collision: nodes continute to transmit their frames

        \begin{itemize}

          \item Entire frame transmission time wasted

        \end{itemize}

    \end{itemize}

  \item CSMA/CD

    \begin{itemize}

      \item Collision detection

        \begin{itemize}

          \item Wired Links: while transmitting, node monitors for the presence of signal energy coming from others nodes

            \begin{itemize}

              \item Signal energy from other nodes detected! Abort transmission

            \end{itemize}

          \item Wireless links: difficult $\to$ use CSMA/CA instead

        \end{itemize}

      \item CSMA/CD reduces the amount of time wasted in collisions

        \begin{itemize}

          \item Transmission aborted on collision detection

        \end{itemize}

    \end{itemize}

  \item Ethernet CSMA/CD Algorithm

    \begin{itemize}

      \item NIC receives packet from network layer, creates frame

      \item NIC senses channel:

        \begin{itemize}

          \item If idle: start frame transmission

          \item If busy: wait until channel idle, then transmit

        \end{itemize}

      \item If NIC transmits entire frame without collision, NIC is done with frame!

      \item If NIC detects another transmission while transmitting, abort and send jam signal

      \item After aborting, NIC enters binary exponential backoff:

        \begin{itemize}

          \item After $n$-th collision, NIC chooses $K$ at random from $\{0,1,2,\ldots,2^n-1\}$

          \item NIC waits $K\cdot512$ bit duration times, returns to step 2

          \item More collisions: longer backoff interval

        \end{itemize}

    \end{itemize}

  \item CSMA/CD Efficiency

    \begin{itemize}

      \item $t_{prop}$ is the maximum propagation delay between two nodes in broadcast link

      \item $t_{trans}$ is the time to transmit maximum-size frame

        $$eff=\frac{1}{1+5t_{prop}/t_{trans}}$$

      \item Efficiency goes to 1 when either propagation goes to zero or transmission goes to infinity

      \item Better performance than ALOHA: and simple and cheap

    \end{itemize}

  \item ``Taking Turns'' MAC Protocols

    \begin{itemize}

      \item Master node ``invites'' slave nodes to transmit in turn

      \item Typically used with ``dumb'' slave devices

      \item Token Passing:

        \begin{itemize}

          \item A small frame, called a token, is passed from one node to the next sequentially

          \item The node that has the token can transmit

          \item Concerns:

            \begin{itemize}

              \item Token overhead

              \item Latency

              \item Single point of failure (token)

            \end{itemize}

        \end{itemize}

    \end{itemize}

  \item Local Area Network (LAN)

    \begin{itemize}

      \item A switch is a link-layer device

        \begin{itemize}

          \item Switches operate at link layer

          \item Switch frames

          \item Do not recognize network-layer addresses

        \end{itemize}

    \end{itemize}

  \item MAC (or LAN or physical or Ethernet) Addresses

    \begin{itemize}

      \item Link-layer address for interface

      \item Function: used ``locally'' to get frame from one interface to another connected interface in the same subnet (in IP-addressing sense)

      \item 48-bit MAC address (for most LANs) burned in NIC ROM, also sometimes software settable

      \item For example: 1A:2F:BB:76:09:AD

      \item Each interface on LAN

        \begin{itemize}

          \item Has unique 48-bit MAC address

          \item Has unique 32-bit/128-bit IP address

        \end{itemize}

      \item MAC address allocation administered by IEEE

      \item Manufacturer buys portion of MAC address space (to assure uniqueness)

      \item Analogy:

        \begin{itemize}

          \item MAC address: like Social Security Number

          \item IP address: like postal address

        \end{itemize}

      \item MAC flat address: portability

        \begin{itemize}

          \item Can move interface from one LAN to another with the same MAC address

          \item Recall IP address is hierarchical and not portable: depends on IP subnet to which node it attached

        \end{itemize}

      \item MAC broadcast address: FF:FF:FF:FF:FF:FF

        \begin{itemize}

          \item Used by a sender that wants all the other interfaces on the LAN to receive and process a frame

        \end{itemize}

    \end{itemize}

  \item ARP: Address Resolution Protocol

    \begin{itemize}

      \item ARP Protocol: to determine MAC address of interface knowing the IP address

      \item ARP table: each IP node (host, router) on LAN has an ARP table

        \begin{itemize}

          \item IP/MAC address mappings for some LAN nodes:

            $$<\text{IP address; MAC address; TTL}>$$

            \begin{itemize}

              \item TTL (Time To Live): time after which address mapping will be forgotten (typically 20 min)

            \end{itemize}

        \end{itemize}

    \end{itemize}

  \item IEEE 802.3: Ethernet

    \begin{itemize}

      \item ``Dominant'' wired LAN technology:

        \begin{itemize}

          \item First widely used LAN technology

          \item Simple and cheap

          \item Kept up with speed race: 10 Mbps — 400 Gbps

          \item Single chip, multiple speeds

        \end{itemize}

    \end{itemize}

  \item Ethernet: Physical Topology

    \begin{itemize}

      \item Bus: popular through mid 90s $\to$ classic Ethernet

        \begin{itemize}
            
          \item All nodes in same collision domain (frames can collide with each other)

        \end{itemize}

      \item Hub-based: popular in late 90s $\to$ classic Ethernet

        \begin{itemize}

          \item A hub is a physical-layer device: input signal sent out on all other interfaces

          \item Hub in center logically operates as a bus: all nodes in same collision domain

        \end{itemize}

      \item Star with a switch: prevails today $\to$ switched Ethernet

        \begin{itemize}

          \item Active link-layer switch in center

          \item Each switch interface is in a different collision domain (frames in different interfaces do not collide with each other)

          \item With full duplex linkes, MAC protocol unnecessary

        \end{itemize}

    \end{itemize}

  \item IEEE 802.3 Ethernet Standards: Link and Physical Layers

    \begin{itemize}

      \item Many different Ethernet standards

        \begin{itemize}

          \item Link layer and physical layer

          \item Common MAC protocol and frame format

          \item Different speeds: 2 Mbps, 10 Mbps, 100 Mbps, 1 Gbps, 10 Gbps, 40 Gbps, 100 Gbps, 400 Gbps, \ldots

          \item Different physical layer and physical media (fiber, cable)

        \end{itemize}

    \end{itemize}

  \item Ethernet: Unreliable, Connectionless

    \begin{itemize}

      \item Ethernet MAC protocol (if needed): unslotted CSMA/CD with binary backoff

      \item Error detection: CRC

      \item Connectionless: no handshaking between sending and receiving NICs

      \item Unreliable: receiving NIC does not send acks or nacks to sending NIC

        \begin{itemize}

          \item CSMA/CD (if needed): collisions are detected in the transmitter and colliding frames retransmitted

          \item Error detection: when an error is detected, the frame is dropped

            \begin{itemize}

              \item Data in dropped frames is recovered end-to-end only if initial sender uses higherlayer reliable data transfer (like TCP), otherwise dropped data is lost

            \end{itemize}

          \item It makes sense as the physical medium has low bit error rate and if collisions are possible, they can be detected (CSMA/CD) and the frame is recovered

        \end{itemize}

    \end{itemize}

\end{itemize}

\end{document}

