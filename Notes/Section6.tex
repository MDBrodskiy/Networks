%%%%%%%%%%%%%%%%%%%%%%%%%%%%%%%%%%%%%%%%%%%%%%%%%%%%%%%%%%%%%%%%%%%%%%%%%%%%%%%%%%%%%%%%%%%%%%%%%%%%%%%%%%%%%%%%%%%%%%%%%%%%%%%%%%%%%%%%%%%%%%%%%%%%%%%%%%%%%%%%%%%
% Written By Michael Brodskiy
% Class: Fundamentals of Networks
% Professor: E. Bernal Mor
%%%%%%%%%%%%%%%%%%%%%%%%%%%%%%%%%%%%%%%%%%%%%%%%%%%%%%%%%%%%%%%%%%%%%%%%%%%%%%%%%%%%%%%%%%%%%%%%%%%%%%%%%%%%%%%%%%%%%%%%%%%%%%%%%%%%%%%%%%%%%%%%%%%%%%%%%%%%%%%%%%%

\include{Includes.tex}

\title{The Link Layer}
\date{\today}
\author{Michael Brodskiy\\ \small Professor: E. Bernal Mor}

\begin{document}

\maketitle

\begin{itemize}

  \item Link Layer: Introduction

    \begin{itemize}

      \item Terminology

        \begin{itemize}

          \item Hosts, routers $\to$ nodes

          \item Communication channels that connect adjacent nodes along communication path $\to$ links

            \begin{itemize}

              \item Wired links

              \item Wireless links

            \end{itemize}

          \item Over a given link, the transmitting node encapsulates the network-layer packet in a link-layer frame

        \end{itemize}

      \item Link layer has responsibility of transferring network-layer packets from one node to a physically adjacent node over a link

    \end{itemize}

  \item Link Layer: Context

    \begin{itemize}

      \item Packets transferred by different link protocols over different links

        \begin{itemize}

          \item WiFi on first link

          \item Ethernet on next link

          \item Etcetera

        \end{itemize}

      \item Each link protocol provides different services

    \end{itemize}

  \item Link Layer Services

    \begin{itemize}

      \item Framing

        \begin{itemize}

          \item Encapsulate packet into frame, adding header and maybe trailer

          \item Addressing: ``MAC'' addresses used in frame headers to identify transmitter/receiver node $\to$ different from IP Address

        \end{itemize}

      \item Link access

        \begin{itemize}

          \item Medium access control (MAC) protocol specifies the rules by which a frame is transmitted onto the link

        \end{itemize}

      \item Flow control

        \begin{itemize}

          \item Pacing between adjacent sending and receiving nodes

        \end{itemize}

      \item Reliable delivery between adjacent nodes

        \begin{itemize}

          \item We learned how to do this already (Transport layer)!

          \item Seldom used on low error rate links, for example: fiber, some twisted pairs

          \item Commonly used on high error rate links, like wireless ones

        \end{itemize}

      \item Error control

        \begin{itemize}

          \item Errors caused by signal attenuation, noise

          \item Error detection: receiver detects presence of errors

            \begin{itemize}

              \item Ask sender for retransmission or drops frame

            \end{itemize}

          \item Error correction: receiver identifies and corrects bit error(s) without resorting to retransmission

        \end{itemize}

    \end{itemize}

  \item Half-Duplex and Full-Duplex Links

    \begin{itemize}

      \item Unidirectional links (Simplex Links)

        \begin{itemize}

          \item Communication occurs in one direction only

        \end{itemize}

      \item Bidirectional links 

        \begin{itemize}

          \item Half-Duplex Link — Communication occurs in both directions, but not at same time

          \item Full-Duplex Link — Communication occurs in both directions at same time

        \end{itemize}

    \end{itemize}

  \item Where is the Link Layer Implemented?

    \begin{itemize}

      \item For the most part, link layer is implemented on a chip called the network adapter, aka a Network Interface Card (NIC)

        \begin{itemize}

          \item The NIC implements Link and Physical layers

          \item \textit{E}.\textit{g}. Ethernet card,  WiFi card or chip

        \end{itemize}

      \item NIC attaches into node’s system buses

      \item Link layer is implemented as a combination of hardware and software

        \begin{itemize}

          \item Hardware: NIC implements most of the functions

          \item Software: activating hardware controller, responds to controller interrupts, etc.

        \end{itemize}

    \end{itemize}

  \item Error Control

    \begin{itemize}

      \item EDC $\to$ Error Detection/Correction bits (redundant bits)

      \item D $\to$ Data protected by error control, may include header fields

      \item Error control is not 100\% reliable

        \begin{itemize}

          \item Error control technique may miss some errors; we want to keep the probability of missing the errors small

          \item Larger EDC field yields better detection and correction

          \item Error correction needs more redundant bits than error detection for same number of errors

        \end{itemize}

      \item Parity Checking

        \begin{itemize}

          \item Single bit parity: detect single bit errors

            \begin{itemize}

              \item Even parity: set parity bit so there is an even number of 1's

              \item Odd parity: set parity bit so there is an odd number of 1's

            \end{itemize}

          \item Two-dimensional bit parity: detect and correct single bit errors

            \begin{itemize}

              \item Even parity: no errors

            \end{itemize}

        \end{itemize}

      \item Cyclic Redundancy Check (CRC)

        \begin{itemize}

          \item $D$: $d$ data bits (given)

          \item $G$: generator, bit pattern of $r+1$ bits where MSB must be 1 $\to$ transmitter and receiver agree on $G$ (given)

          \item $R$: $r$ CRC bits, redundant bits

          \item Transmitter: choose $R$, such that $\langle D,R\rangle$ is exactly divisible by $G$ (modulo-2 arithmetic) $\to D\cdot2^r\text{XOR}R=nG$

          \item Receiver: knows $G$ and divides $\langle D,R\rangle$ by $G\to$ non-zero remainder: error detected!

            \begin{itemize}

              \item Can detect all burst errors less than $r+1$ bits

            \end{itemize}

          \item More powerful error-detection technique: widely used in practice (Ethernet,  WiFi)

        \end{itemize}

    \end{itemize}

\end{itemize}

\end{document}

