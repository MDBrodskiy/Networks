%%%%%%%%%%%%%%%%%%%%%%%%%%%%%%%%%%%%%%%%%%%%%%%%%%%%%%%%%%%%%%%%%%%%%%%%%%%%%%%%%%%%%%%%%%%%%%%%%%%%%%%%%%%%%%%%%%%%%%%%%%%%%%%%%%%%%%%%%%%%%%%%%%%%%%%%%%%%%%%%%%%
% Written By Michael Brodskiy
% Class: Fundamentals of Networks
% Professor: E. Bernal Mor
%%%%%%%%%%%%%%%%%%%%%%%%%%%%%%%%%%%%%%%%%%%%%%%%%%%%%%%%%%%%%%%%%%%%%%%%%%%%%%%%%%%%%%%%%%%%%%%%%%%%%%%%%%%%%%%%%%%%%%%%%%%%%%%%%%%%%%%%%%%%%%%%%%%%%%%%%%%%%%%%%%%

\documentclass[12pt]{article} 
\usepackage{alphalph}
\usepackage[utf8]{inputenc}
\usepackage[russian,english]{babel}
\usepackage{titling}
\usepackage{amsmath}
\usepackage{graphicx}
\usepackage{enumitem}
\usepackage{amssymb}
\usepackage[super]{nth}
\usepackage{everysel}
\usepackage{ragged2e}
\usepackage{geometry}
\usepackage{multicol}
\usepackage{fancyhdr}
\usepackage{cancel}
\usepackage{siunitx}
\usepackage{physics}
\usepackage{tikz}
\usepackage{mathdots}
\usepackage{yhmath}
\usepackage{cancel}
\usepackage{color}
\usepackage{array}
\usepackage{multirow}
\usepackage{gensymb}
\usepackage{tabularx}
\usepackage{extarrows}
\usepackage{booktabs}
\usepackage{lastpage}
\usepackage{float}
\usetikzlibrary{fadings}
\usetikzlibrary{patterns}
\usetikzlibrary{shadows.blur}
\usetikzlibrary{shapes}

\geometry{top=1.0in,bottom=1.0in,left=1.0in,right=1.0in}
\newcommand{\subtitle}[1]{%
  \posttitle{%
    \par\end{center}
    \begin{center}\large#1\end{center}
    \vskip0.5em}%

}
\usepackage{hyperref}
\hypersetup{
colorlinks=true,
linkcolor=blue,
filecolor=magenta,      
urlcolor=blue,
citecolor=blue,
}


\title{Fundamentals of Networks}
\date{\today}
\author{Michael Brodskiy\\ \small Professor: E. Bernal Mor}

\begin{document}

\maketitle

\begin{itemize}

  \item What is a Network?

    \begin{itemize}

      \item A collection of devices (aka nodes) interconnected by different types of links, which allow devices to communicate at a distance in order to support diverse applications

      \item Devices:

        \begin{itemize}

          \item End systems or terminal nodes: computer (desktop, laptop), cell-phone, tablet, car, sensor, and virtually almost anything

          \item Intermediate nodes: modem, repeater, hub, switch, router, base station, etc.

        \end{itemize}

      \item Links:

        \begin{itemize}

          \item Wired links: fiber, copper, etc.

          \item Wireless links: electromagnetic waves (radio, microwave, terahertz band, infra-red, etc.), acoustic waves (ultra-sounds), etc.

        \end{itemize}

      \item Applications:

        \begin{itemize}

          \item E-mail, instant messaging, web browsing, multimedia streaming, etc.

        \end{itemize}

    \end{itemize}

  \item A wireless sensor network

    \begin{itemize}

      \item Many applications:

        \begin{itemize}

          \item Military applications: battlefield surveillance, nuclear, biological, and chemical attack prevention, etc.

          \item Environmental applications: tracking birds, smart irrigation, earth monitoring, etc.

          \item Health applications: health telemonitoring, drug administration tracking, etc.

        \end{itemize}

    \end{itemize}

  \item Network Types (classified by size):

    \begin{itemize}

      \item PAN (Personal Area Networks) — Bluetooth, USB, etc.

        \begin{itemize}

          \item Range of a person

        \end{itemize}

      \item LAN (Local Area Networks) — WiFi, Ethernet, etc.

        \begin{itemize}

          \item Range of a single building: a home, office, or factory

        \end{itemize}

      \item MAN (Metropolitan Area Networks) — WiMax, cable, etc.

        \begin{itemize}

          \item Range of a city

        \end{itemize}

      \item WAN (Wide Area Networks) — Cellular, landline telephone, etc.

        \begin{itemize}

          \item Range of an entire country or continent

        \end{itemize}

      \item Satellite

    \end{itemize}

  \item Thus, the internet is a ``network of networks''

    \begin{itemize}

      \item Billions of connected computing devices:

        \begin{itemize}

          \item Hosts = end systems

          \item Running network applications at Internet's ``edge''

        \end{itemize}

      \item Packet switches: forward packets (chunks of data)

          \begin{itemize}

            \item Routers, switches, etc.

          \end{itemize}

        \item Communication links

          \begin{itemize}

            \item Fiber, copper, radio, satellite

            \item Transmission rate: link capacity (bps)

          \end{itemize}

        \item Networks

          \begin{itemize}

            \item Managed by organization

          \end{itemize}

        \item Interconnected ISPs (Internet Service Providers)

    \end{itemize}

  \item Protocols are everywhere

    \begin{itemize}

      \item Control sending, receiving of messages

      \item Examples: HTTP (Web), streaming video, Skype, TCP, IP, WiFi, 4G, 5G, Ethernet

    \end{itemize}

  \item Internet Standardization

    \begin{itemize}

      \item IETF: Internet Engineering Task Force

        \begin{itemize}

          \item RFC: Request for Comments

        \end{itemize}

      \item IEEE: Institute of Electrical and Electronics Engineers

        \begin{itemize}

          \item IEEE 802.3, IEEE 802.11

        \end{itemize}

    \end{itemize}

  \item Infrastructure that provides services to applications:

    \begin{itemize}

      \item Web, streaming video, multimedia teleconferencing, e-mail, games, e-commerce, social media, interconnected appliances

    \end{itemize}
    
  \item Provides programming interface to distributed applications:

    \begin{itemize}

      \item ``Hooks'' allowing sending/receiving applications ``connect'' to, use Internet transport service

      \item Provides service options, analogous to postal service

    \end{itemize}

  \item Protocols

    \begin{itemize}

      \item For humans, an example is language (we have phonetics, grammar, etc.)

      \item All communication activity in Internet governed by protocols

      \item Sample definition: Protocols define the format, order of messages sent and received among network entities, and actions taken on message transmission and receipt.

    \end{itemize}

  \item A closer look at internet structure:

    \begin{itemize}

      \item Edge of the network:

        \begin{itemize}

          \item Hosts: clients and servers

          \item Servers often in data centers

        \end{itemize}

      \item Access networks:

        \begin{itemize}

          \item Wired, wireless communication links

        \end{itemize}

      \item Network core:

        \begin{itemize}

          \item Interconnected routers

          \item Network of networks

        \end{itemize}

    \end{itemize}

  \item Physical Media

    \begin{itemize}

      \item Bit: unit of information that is carried by the signal that propagates between transmitter and receiver

      \item Physical link: what lies between transmitter and receiver

      \item Types of media:

        \begin{itemize}

          \item Guided media: signals propagate solid media (e.g.\ copper, fiber, coaxial)

            \begin{itemize}

              \item Twisted pair (TP) — Two insulated copper wires twisted together in a helical form (The signals are usually carried as the difference in voltage between the two wires in the pair to increase robustness against noise).

              \item Coaxial cable — Two concentric cooper conductors, with bidirectional capabilities. Longer distances at higher data transmission rates than twisted pairs. A broadband system; that is, multiple frequency channels on cable.

              \item Fiber optic cable — Glass fiber carrying light pulses, each pulse a bit. High-speed operation, with point-to-point transmissions ranging from 10's-100's Gbps. Very low error rate because it is immune to electromagnetic noise, with repeaters spaced far apart. Downside: expensive and fragile

            \end{itemize}

          \item Unguided media: signals propagate freely (no physical wire, like a radio)

            \begin{itemize}

              \item Signal can be carried in different ways — electromagnetic waves (most commonly used), acoustic waves (typically underwater), magnetic-induction (e.g.\ Near Field Communications)

              \item Propagation environment effects — reflection, obstruction by objects, interference

            \end{itemize}

        \end{itemize}

    \end{itemize}

  \item Network types classified by switching technology — there are two fundamental approaches to moving data through a network of links

    \begin{itemize}

      \item Circuit Switching

        \begin{itemize}

          \item Resources needed along a path (e.g.\ link transmission rate) to provide for communication between hosts are reserved before the transmission starts

          \item This reservation defines the path followed by data and guarantees the communication

          \item The resources are reserved for the duration of the communication session between the hosts

        \end{itemize}

      \item Packet Switching

        \begin{itemize}

          \item A message is divided into smaller data units or packets, which are transmitted over the network using the resources demanded

          \item No reservation of resources

          \item Different packets in the same ``information stream'' may traverse different paths and suffer different delays

        \end{itemize}

    \end{itemize}

  \item The Network Core

    \begin{itemize}

      \item The network core of the Internet is a meash of interconnected packet switches

      \item Thus, the Internet is a packet-switched network

      \item Packet switching: hosts break host messages into packets

        \begin{itemize}

          \item Packets are forwarded from one router to the next across links on path from source to destination

          \item Each packet is transmitted at full link capacity in each link

          \item No reservation $\rightarrow$ A packet may have to wait (that is, queue on a router)

        \end{itemize}

      \item Host sending function

        \begin{enumerate}

          \item Takes application message

          \item Breaks into smaller chunks or packets, of length $L$ bits

          \item Transmits packet into access network at transmission rate $R$ (bits/sec)

            \begin{itemize}
                
              \item Full link transmission rate: link capacity (aka link bandwidth)

            \end{itemize}

          \item Packet transmission delay = Time needed to transmit $L$-bit packet into link = $\frac{L}{R}$

        \end{enumerate}

      \item Router Function

        \begin{itemize}
            
          \item Store and forward — entire packet must arrive at input link of router before it can be forwarded to output link and then transmitted on next link

          \item Transmission delay — takes $L$/$R$ seconds to transmit (push out) $L$-bit packet into output link at $R$ bps

          \item End-to-end transmission delay — $2L$/$R$, ignoring other types of delays (more on delay shortly)

        \end{itemize}

    \end{itemize}

  \item Circuit Switching

    \begin{itemize}

      \item End-to-end resources reserved for, and allocated to communication between source and destination

      \item Commonly used in traditional telephone networks

      \item Usually, the communication is referred to as a ``call''

      \item Before the sender can send the information, the network must establish a connection between the sender and the receiver (resource reservation)

        \begin{itemize}

          \item Switches on the path between the sender and receiver maintain connection state

        \end{itemize}

    \end{itemize}

  \item Internet Structure

    \begin{itemize}

      \item Hosts connect to Internet via access to Internet Service Provider (ISPs)

        \begin{itemize}

          \item Residential, enterprise (company, university, commercial) ISPs

        \end{itemize}

      \item Access ISPs in turn must be interconnected

        \begin{itemize}

          \item So that any two hosts can send packets to each other

        \end{itemize}

      \item Resulting network of networks is very complex

        \begin{itemize}

          \item Evolution was driven by economics and national policies

        \end{itemize}

    \end{itemize}

  \item Nodal Delay

    \begin{itemize}

      \item Four types — Nodal processing, transmission, queueing, and propagation

      \item $d_{nodal}=d_{proc}+d_{queue}+d_{trans}+d_{prop}$

      \item Nodal Processing ($d_{proc}$)

        \begin{itemize}

          \item Check bit errors, determine output link, \ldots

          \item Typically $<1[\si{\micro\second}]$ or less

        \end{itemize}

      \item Queuing Delay ($d_{queue}$)

        \begin{itemize}

          \item Time waiting at output link for transmission

          \item Depends on congestion level of router

          \item Can vary from packet to packet

          \item Each output link has its own buffer (aka queue)

          \item Buffers have finite memory space

          \item Arrival rate (bps) to output link in router exceeds transmission rate, $R$ bps, of this link temporarily

        \end{itemize}

      \item Transmission Delay ($d_{trans}$)

        \begin{itemize}

          \item $L$ packet length (bits)

          \item $R$ link transmission rate (bps)

          \item $d_{trans}=L/R$

        \end{itemize}

      \item Propagation Delay ($d_{prop}$)

        \begin{itemize}

          \item $d$ length of physical link

          \item $s$ propagation speed ($2\cdot10^{8}$ to $3\cdot10^8$ meters per second)

          \item $d_{prop}=d/s$

        \end{itemize}

    \end{itemize}

    \item Packet Queuing Delay

    \begin{itemize}

      \item $R$ link capacity (bps)

      \item $L$ packet length (bits per packet)

      \item $a$ average packet arrival rate (packets per second)

      \item The traffic intensity is $(La)/R$

        \begin{itemize}

          \item When $(La)/R\approx 0$, average queueing delay small

          \item When $(La)/R=1$, average queueing delay large

          \item When $(La)/R>1$, ``work'' arriving is more than can be serviced — average delay is infinite

        \end{itemize}

    \end{itemize}

\end{itemize}

\end{document}

