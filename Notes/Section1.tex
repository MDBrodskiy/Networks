%%%%%%%%%%%%%%%%%%%%%%%%%%%%%%%%%%%%%%%%%%%%%%%%%%%%%%%%%%%%%%%%%%%%%%%%%%%%%%%%%%%%%%%%%%%%%%%%%%%%%%%%%%%%%%%%%%%%%%%%%%%%%%%%%%%%%%%%%%%%%%%%%%%%%%%%%%%%%%%%%%%
% Written By Michael Brodskiy
% Class: Fundamentals of Networks
% Professor: E. Bernal Mor
%%%%%%%%%%%%%%%%%%%%%%%%%%%%%%%%%%%%%%%%%%%%%%%%%%%%%%%%%%%%%%%%%%%%%%%%%%%%%%%%%%%%%%%%%%%%%%%%%%%%%%%%%%%%%%%%%%%%%%%%%%%%%%%%%%%%%%%%%%%%%%%%%%%%%%%%%%%%%%%%%%%

\include{Includes.tex}

\title{Fundamentals of Networks}
\date{\today}
\author{Michael Brodskiy\\ \small Professor: E. Bernal Mor}

\begin{document}

\maketitle

\begin{itemize}

  \item What is a Network?

    \begin{itemize}

      \item A collection of devices (aka nodes) interconnected by different types of links, which allow devices to communicate at a distance in order to support diverse applications

      \item Devices:

        \begin{itemize}

          \item End systems or terminal nodes: computer (desktop, laptop), cell-phone, tablet, car, sensor, and virtually almost anything

          \item Intermediate nodes: modem, repeater, hub, switch, router, base station, etc.

        \end{itemize}

      \item Links:

        \begin{itemize}

          \item Wired links: fiber, copper, etc.

          \item Wireless links: electromagnetic waves (radio, microwave, terahertz band, infra-red, etc.), acoustic waves (ultra-sounds), etc.

        \end{itemize}

      \item Applications:

        \begin{itemize}

          \item E-mail, instant messaging, web browsing, multimedia streaming, etc.

        \end{itemize}

    \end{itemize}

\end{itemize}

\end{document}

