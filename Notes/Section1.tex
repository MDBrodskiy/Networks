%%%%%%%%%%%%%%%%%%%%%%%%%%%%%%%%%%%%%%%%%%%%%%%%%%%%%%%%%%%%%%%%%%%%%%%%%%%%%%%%%%%%%%%%%%%%%%%%%%%%%%%%%%%%%%%%%%%%%%%%%%%%%%%%%%%%%%%%%%%%%%%%%%%%%%%%%%%%%%%%%%%
% Written By Michael Brodskiy
% Class: Fundamentals of Networks
% Professor: E. Bernal Mor
%%%%%%%%%%%%%%%%%%%%%%%%%%%%%%%%%%%%%%%%%%%%%%%%%%%%%%%%%%%%%%%%%%%%%%%%%%%%%%%%%%%%%%%%%%%%%%%%%%%%%%%%%%%%%%%%%%%%%%%%%%%%%%%%%%%%%%%%%%%%%%%%%%%%%%%%%%%%%%%%%%%

\include{Includes.tex}

\title{Fundamentals of Networks}
\date{\today}
\author{Michael Brodskiy\\ \small Professor: E. Bernal Mor}

\begin{document}

\maketitle

\begin{itemize}

  \item What is a Network?

    \begin{itemize}

      \item A collection of devices (aka nodes) interconnected by different types of links, which allow devices to communicate at a distance in order to support diverse applications

      \item Devices:

        \begin{itemize}

          \item End systems or terminal nodes: computer (desktop, laptop), cell-phone, tablet, car, sensor, and virtually almost anything

          \item Intermediate nodes: modem, repeater, hub, switch, router, base station, etc.

        \end{itemize}

      \item Links:

        \begin{itemize}

          \item Wired links: fiber, copper, etc.

          \item Wireless links: electromagnetic waves (radio, microwave, terahertz band, infra-red, etc.), acoustic waves (ultra-sounds), etc.

        \end{itemize}

      \item Applications:

        \begin{itemize}

          \item E-mail, instant messaging, web browsing, multimedia streaming, etc.

        \end{itemize}

    \end{itemize}

  \item A wireless sensor network

    \begin{itemize}

      \item Many applications:

        \begin{itemize}

          \item Military applications: battlefield surveillance, nuclear, biological, and chemical attack prevention, etc.

          \item Environmental applications: tracking birds, smart irrigation, earth monitoring, etc.

          \item Health applications: health telemonitoring, drug administration tracking, etc.

        \end{itemize}

    \end{itemize}

  \item Network Types (classified by size):

    \begin{itemize}

      \item PAN (Personal Area Networks) — Bluetooth, USB, etc.

        \begin{itemize}

          \item Range of a person

        \end{itemize}

      \item LAN (Local Area Networks) — WiFi, Ethernet, etc.

        \begin{itemize}

          \item Range of a single building: a home, office, or factory

        \end{itemize}

      \item MAN (Metropolitan Area Networks) — WiMax, cable, etc.

        \begin{itemize}

          \item Range of a city

        \end{itemize}

      \item WAN (Wide Area Networks) — Cellular, landline telephone, etc.

        \begin{itemize}

          \item Range of an entire country or continent

        \end{itemize}

      \item Satellite

    \end{itemize}

  \item Thus, the internet is a ``network of networks''

    \begin{itemize}

      \item Billions of connected computing devices:

        \begin{itemize}

          \item Hosts = end systems

          \item Running network applications at Internet's ``edge''

        \end{itemize}

      \item Packet switches: forward packets (chunks of data)

          \begin{itemize}

            \item Routers, switches, etc.

          \end{itemize}

        \item Communication links

          \begin{itemize}

            \item Fiber, copper, radio, satellite

            \item Transmission rate: link capacity (bps)

          \end{itemize}

        \item Networks

          \begin{itemize}

            \item Managed by organization

          \end{itemize}

        \item Interconnected ISPs (Internet Service Providers)

    \end{itemize}

  \item Protocols are everywhere

    \begin{itemize}

      \item Control sending, receiving of messages

      \item Examples: HTTP (Web), streaming video, Skype, TCP, IP, WiFi, 4G, 5G, Ethernet

    \end{itemize}

  \item Internet Standardization

    \begin{itemize}

      \item IETF: Internet Engineering Task Force

        \begin{itemize}

          \item RFC: Request for Comments

        \end{itemize}

      \item IEEE: Institute of Electrical and Electronics Engineers

        \begin{itemize}

          \item IEEE 802.3, IEEE 802.11

        \end{itemize}

    \end{itemize}

  \item Infrastructure that provides services to applications:

    \begin{itemize}

      \item Web, streaming video, multimedia teleconferencing, e-mail, games, e-commerce, social media, interconnected appliances

    \end{itemize}
    
  \item Provides programming interface to distributed applications:

    \begin{itemize}

      \item ``Hooks'' allowing sending/receiving applications ``connect'' to, use Internet transport service

      \item Provides service options, analogous to postal service

    \end{itemize}

  \item Protocols

    \begin{itemize}

      \item For humans, an example is language (we have phonetics, grammar, etc.)

      \item All communication activity in Internet governed by protocols

      \item Sample definition: Protocols define the format, order of messages sent and received among network entities, and actions taken on message transmission and receipt.

    \end{itemize}

  \item A closer look at internet structure:

    \begin{itemize}

      \item Edge of the network:

        \begin{itemize}

          \item Hosts: clients and servers

          \item Servers often in data centers

        \end{itemize}

      \item Access networks:

        \begin{itemize}

          \item Wired, wireless communication links

        \end{itemize}

      \item Network core:

        \begin{itemize}

          \item Interconnected routers

          \item Network of networks

        \end{itemize}

    \end{itemize}

  \item Physical Media

    \begin{itemize}

      \item Bit: unit of information that is carried by the signal that propagates between transmitter and receiver

      \item Physical link: what lies between transmitter and receiver

      \item Types of media:

        \begin{itemize}

          \item Guided media: signals propagate solid media (e.g.\ copper, fiber, coaxial)

            \begin{itemize}

              \item Twisted pair (TP) — Two insulated copper wires twisted together in a helical form (The signals are usually carried as the difference in voltage between the two wires in the pair to increase robustness against noise).

              \item Coaxial cable — Two concentric cooper conductors, with bidirectional capabilities. Longer distances at higher data transmission rates than twisted pairs. A broadband system; that is, multiple frequency channels on cable.

              \item Fiber optic cable — Glass fiber carrying light pulses, each pulse a bit. High-speed operation, with point-to-point transmissions ranging from 10's-100's Gbps. Very low error rate because it is immune to electromagnetic noise, with repeaters spaced far apart. Downside: expensive and fragile

            \end{itemize}

          \item Unguided media: signals propagate freely (no physical wire, like a radio)

            \begin{itemize}

              \item Signal can be carried in different ways — electromagnetic waves (most commonly used), acoustic waves (typically underwater), magnetic-induction (e.g.\ Near Field Communications)

              \item Propagation environment effects — reflection, obstruction by objects, interference

            \end{itemize}

        \end{itemize}

    \end{itemize}

\end{itemize}

\end{document}

