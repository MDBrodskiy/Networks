%%%%%%%%%%%%%%%%%%%%%%%%%%%%%%%%%%%%%%%%%%%%%%%%%%%%%%%%%%%%%%%%%%%%%%%%%%%%%%%%%%%%%%%%%%%%%%%%%%%%%%%%%%%%%%%%%%%%%%%%%%%%%%%%%%%%%%%%%%%%%%%%%%%%%%%%%%%%%%%%%%%
% Written By Michael Brodskiy
% Class: Fundamentals of Networks
% Professor: E. Bernal Mor
%%%%%%%%%%%%%%%%%%%%%%%%%%%%%%%%%%%%%%%%%%%%%%%%%%%%%%%%%%%%%%%%%%%%%%%%%%%%%%%%%%%%%%%%%%%%%%%%%%%%%%%%%%%%%%%%%%%%%%%%%%%%%%%%%%%%%%%%%%%%%%%%%%%%%%%%%%%%%%%%%%%

\documentclass[12pt]{article} 
\usepackage{alphalph}
\usepackage[utf8]{inputenc}
\usepackage[russian,english]{babel}
\usepackage{titling}
\usepackage{amsmath}
\usepackage{graphicx}
\usepackage{enumitem}
\usepackage{amssymb}
\usepackage[super]{nth}
\usepackage{everysel}
\usepackage{ragged2e}
\usepackage{geometry}
\usepackage{multicol}
\usepackage{fancyhdr}
\usepackage{cancel}
\usepackage{siunitx}
\usepackage{physics}
\usepackage{tikz}
\usepackage{mathdots}
\usepackage{yhmath}
\usepackage{cancel}
\usepackage{color}
\usepackage{array}
\usepackage{multirow}
\usepackage{gensymb}
\usepackage{tabularx}
\usepackage{extarrows}
\usepackage{booktabs}
\usepackage{lastpage}
\usepackage{float}
\usetikzlibrary{fadings}
\usetikzlibrary{patterns}
\usetikzlibrary{shadows.blur}
\usetikzlibrary{shapes}

\geometry{top=1.0in,bottom=1.0in,left=1.0in,right=1.0in}
\newcommand{\subtitle}[1]{%
  \posttitle{%
    \par\end{center}
    \begin{center}\large#1\end{center}
    \vskip0.5em}%

}
\usepackage{hyperref}
\hypersetup{
colorlinks=true,
linkcolor=blue,
filecolor=magenta,      
urlcolor=blue,
citecolor=blue,
}


\title{Wireless Networks}
\date{\today}
\author{Michael Brodskiy\\ \small Professor: E. Bernal Mor}

\begin{document}

\maketitle

\begin{itemize}

  \item Wireless and Mobile Networks: Context

    \begin{itemize}

      \item More wireless (mobile) phone subscribers than fixed (wired) phone subscribers (10-to-1 in 2029)

      \item More mobile-broadband connected devices than fixed-broadband connected devices (5-1 in 2019)

        \begin{itemize}

          \item 4G/5G cellular networks now embracing Internet protocol stack, including SDN

        \end{itemize}

      \item Two important (but different) challenges

        \begin{itemize}

          \item Wireless: communication over wireless link

          \item Mobility: handling the mobile user who changes point of attachment to network

        \end{itemize}

    \end{itemize}

  \item Wireless Hosts

    \begin{itemize}

      \item Laptop, smartphone, IoT device, etc.

      \item Run applications

      \item May be stationary (non-mobile) or mobile

        \begin{itemize}

          \item Wireless does not always mean mobility

        \end{itemize}

    \end{itemize}

  \item Base Station

    \begin{itemize}

      \item Key element that connects the wireless network to wired networks through the backbone link

      \item Relay — responsible for sending packets between wired networks and wireless host(s) in its ``area''

        \begin{itemize}

          \item For examples, cell towers, IEEE 802.11, access points

        \end{itemize}

    \end{itemize}

  \item Network Infrastructure

    \begin{itemize}

      \item Larger network with which a wireless host may wish to communicate

      \item Backbone link: connects base station to network infrastructure

    \end{itemize}

  \item Infrastructure Mode

    \begin{itemize}

      \item Each node is associated to the base station

      \item Base station connects wireless hosts into the wired network

      \item Handoff or handover: mobile node changes base station providing connection into wired network (without losing connectivity)

    \end{itemize}

  \item Operating Modes of a Wireless Network

    \begin{itemize}

      \item Ad Hoc Mode

        \begin{itemize}

          \item No base stations

          \item No larger network infrastructure to connect

          \item Nodes can only transmit to other nodes within link coverage

          \item Nodes organize themselves into a network: route among themselves

          \item Development of protocols is challenging

        \end{itemize}

    \end{itemize}

  \item Wireless Link Characteristics

    \begin{itemize}

      \item Decreased signal strength: radio signal attenuates as it propagates through matter (path loss)

      \item Interference from other sources: some wireless network frequencies (like 2.4 GHz) are shared by many devices (like WiFi, Bluetooth, garage openers, motors, etc), which cause interference

      \item Multipath propagation: radio signal reflects off objects and ground, arriving at destination may create different copies of the signal at slightly different times (multipath fading)

      \item Moreover, nodes in the wireless link (broadcast link) do not receive the same signal

    \end{itemize}

  \item WiFi: IEEE 802.11 Wireless LAN

    \begin{itemize}

      \item Many different 802.11 Standards:

        \begin{itemize}

          \item Link layer and physical layer

          \item Most of the IEEE 802.11 standards have infrastructure mode and ad hoc mode network versions

          \item Common MAC protocol

          \item Common frame format

          \item All offer connectionless reliable service with positive ACK and stop and wait at the link layer

          \item Different data rates

          \item All (but one) operate in unlicensed spectrum $\to$ ISM bands (Industrial, Scientific, and Medical bands)

            \begin{itemize}

              \item Open frequency bands for non-exclusive usage (no license required)

            \end{itemize}

        \end{itemize}

    \end{itemize}

  \item Infrastructure Mode

    \begin{itemize}

      \item Base station is the access point (AP)

      \item Wireless host communicates with AP

      \item Service set identifier (SSID): AP identifier

      \item Basic Service Set (BSS) contains:

        \begin{itemize}

          \item Wireless hosts

          \item AP

        \end{itemize}

    \end{itemize}

  \item IEEE 802.11 MAC Protocol: CSMA/CA

    \begin{itemize}

      \item IEEE 802.11 Sender

        \begin{enumerate}

          \item If sense channel idle for DIFs, then:

            \begin{itemize}

              \item Transmit entire frame (no CD)

            \end{itemize}

          \item If sense channel is busy, then:

            \begin{itemize}

              \item Wait until channel is idle for DIFS

              \item Start random backoff time using binary exponential backoff

              \item Timer counts down only when channel idle

            \end{itemize}

          \item Transmit when timer expires (only occurs when channel idle)

            \begin{itemize}

              \item Wait for ACK

            \end{itemize}

          \item If ACK received and another frame to send, begin CSMA/CA in step 2

          \item If no ACK, increase random backoff interval, repeat step 2

        \end{enumerate}

      \item IEEE 802.11 Receiver

        \begin{itemize}

          \item If frame received OK (no collision and no errors detected)

            \begin{itemize}

              \item Return ACK after SIFS

            \end{itemize}

        \end{itemize}

      \item Inter-frame spaces

        \begin{itemize}

          \item Used to give access priority to different transmissions:

            \begin{itemize}

              \item SIFS (Short Inter-Frame Space)

              \item DIFS (Distributed Inter-Frame Space)

            \end{itemize}

          \item The smaller interface space, the higher priority

        \end{itemize}

    \end{itemize}

  \item IEEE 802.11: RTS/CTS

    \begin{itemize}

      \item IEEE 802.11 MAC protocol includes an optionalreservation scheme to deal with hidden terminal problem

      \item Idea: sender “reserves” channel for data frames using small reservation packets

        \begin{itemize}

          \item Using CSMA/CA, sender transmits a short RTS (Request-To-Send) frame to receiver indicating the total time required to transmit the data frame

          \item After SIFS, receiver broadcasts CTS (Clear-To-Send) in response to RTS, including the reserved time

            \begin{itemize}

              \item After receiving the CTS, sender waits SIFS and transmits data frame

              \item Nodes that create collisions in receiver hear CTS and defer transmissions for the reserved time

            \end{itemize}

        \end{itemize}

      \item RTS may still collide with other RTS, but RTS are short frames

        \begin{itemize}

          \item RTS/CTS only worth it for transmission of long data frames

          \item Each node sets an RTS threshold $\to$ If frame is longer than threshold, then use RTS/CTS

        \end{itemize}

    \end{itemize}

  \item IEEE 802.11: Reliable, Connectionless

    \begin{itemize}

      \item MAC Protocol: CSMA/CA

      \item Error Detection in link layer: CRC

      \item Connectionless: no handshaking between sending and receiving NICs

      \item Reliable: receiving NIC sends positive ACKs to sending NIC if frame is received correctly

        \begin{itemize}

          \item Implements stop and wait

          \item CSMA/CA: collisions can still happen and can not be detected

            \begin{itemize}

              \item Transmitter needs acknowledgements to know if frame was received without collisions

            \end{itemize}

          \item Error detection in link layer to detect bit errors introduced by the physical medium and physical layer

            \begin{itemize}

              \item Transmitter needs acknowledgements to know if frame was received correctly or otherwise recover from the error in the link

            \end{itemize}

        \end{itemize}

    \end{itemize}

  \item Wireless, Mobility: Impact on Higher Layer Protocols

    \begin{itemize}

      \item Logically, impact on higher layers should be minimal

        \begin{itemize}

          \item Best effort service model remains unchanges

          \item TCP and UDP can (and do) run over wireless, mobile

        \end{itemize}

      \item Performance-wise:

        \begin{itemize}

          \item Packet loss/delay due to bit-errors (discarded packets, delays for link-layer retransmissions), and handover loss

            \begin{itemize}

              \item TCP interprets loss as congestion, will decrease congestion window un-necessarily

              \item Delay impairments for real-time traffic

            \end{itemize}

          \item Link capacity is a scarce resource for wireless links

        \end{itemize}

    \end{itemize}

\end{itemize}

\end{document}

